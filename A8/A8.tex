% --------------------------------------------------------------
% This is all preamble stuff that you don't have to worry about.
% Head down to where it says "Start here"
% --------------------------------------------------------------
 
\documentclass[12pt]{article}
 
\usepackage[margin=1in]{geometry} 
\usepackage{amsmath,amsthm,amssymb}
\usepackage{bbm}
\usepackage{graphicx}


\usepackage{fancyhdr}
\usepackage[us,12hr]{datetime} % `us' makes \today behave as usual in TeX/LaTeX
\fancypagestyle{plain}{
\fancyhf{}
\rfoot{Compiled on {\ddmmyyyydate\today} at \currenttime}
\lfoot{Page \thepage}
\renewcommand{\headrulewidth}{0pt}}
\pagestyle{plain}

 
\newcommand{\N}{\mathbb{N}}
\newcommand{\Z}{\mathbb{Z}}
\newcommand{\R}{\mathbb{R}}
\newcommand{\Q}{\mathbb{Q}}
\newcommand{\C}{\mathbb{C}}

 
\newenvironment{theorem}[2][Theorem]{\begin{trivlist}
\item[\hskip \labelsep {\bfseries #1}\hskip \labelsep {\bfseries #2.}]}{\end{trivlist}}
\newenvironment{lemma}[2][Lemma]{\begin{trivlist}
\item[\hskip \labelsep {\bfseries #1}\hskip \labelsep {\bfseries #2.}]}{\end{trivlist}}
\newenvironment{exercise}[2][Exercise]{\begin{trivlist}
\item[\hskip \labelsep {\bfseries #1}\hskip \labelsep {\bfseries #2.}]}{\end{trivlist}}
\newenvironment{problem}[2][Problem]{\begin{trivlist}
\item[\hskip \labelsep {\bfseries #1}\hskip \labelsep {\bfseries #2.}]}{\end{trivlist}}
\newenvironment{question}[2][Question]{\begin{trivlist}
\item[\hskip \labelsep {\bfseries #1}\hskip \labelsep {\bfseries #2.}]}{\end{trivlist}}
\newenvironment{corollary}[2][Corollary]{\begin{trivlist}
\item[\hskip \labelsep {\bfseries #1}\hskip \labelsep {\bfseries #2.}]}{\end{trivlist}}
 
\begin{document}

% --------------------------------------------------------------
%                         Start here
% --------------------------------------------------------------
 
\title{Math 223 \\
Assignment 8
}
\author{Quinn Neumiiller} 

\maketitle

% --------------------------------------------------------------
%                         Q1
% --------------------------------------------------------------
\begin{question}{1}
Find $GCD(f(x), g(x))$, and write it as a linear combination of $f(x)$ and $g(x)$ in the
indicated polynomial ring $F[x]$.

\begin{question}{1a}$f(x) = x^8 + x^7 + x^6 + x^4 + x^3 + x + 1$ and $g(x) = x^6 + x^5 + x^3 + x$ in $\Z_{2}[x]$.
Euclidean Algorithm:

$x^8 + x^7 + x^6 + x^4 + x^3 + x + 1 = (x^6 + x^5 + x^3 + x)(x^2 + 1) + (x^4 + x^3 + 1)$

$x^6 + x^5 + x^3 + x = (x^4 + x^3 + 1)(x^2) + (x^3 + x^2 + x)$

$x^4 + x^3 + 1 = (x^3 + x^2 + x)(x) + (x^2+1)$ %

$x^3 + x^2 + x = (x^2+1)(x+1) + (1)$ %

$x^2 + 1 = (1)(x+1) + (0)$ %

Let $a = x^2+1$, $b = x^3 + x^2 + x$, $c = x^4 + x^3 + 1$

Back Substitute: 

$1 = b(-a)(x+1)$

$1 = b(-((c)(-b)(x)))(x+1)$
$1 = b(-c)(b)(-x)(x+1)$
$1 = 2b(-c)(-x)(x+1)$
$1 = 2(g(x))(-(c)(x^2))(-c)(-x)(x+1)$
$1 = 2(g(x))(x^2)(-2c)(-x)(x+1)$
$1 = 2(g(x))(x^2)(-2(f(x)-g(x)))(-x)(x+1)$
$1 = 2g(x)(x^2)(-2(f(x)-g(x)))(-x)(x+1)$


\end{question}

\begin{question}{1b}$f(x) = x^5 + x^4 + 2x^2 + 5x + 4$ and $g(x) = x^2 + 3x + 1$ in $\Z_{7}[x]$
Euclidean Algorithm:

$f(x) = g(x)(x^3 + 5x^2 + 5x + 3) + 5(x + 3)$ %pulled 5 so 5(x+3)

$g(x) = (x + 3)(x) + (1)$

Back Substitute: 

$1 = g(x) - (x + 3)(x)$

$1 = g(x) - 3[f(x) - g(x)(x^3 + 5x^2 + 5x + 3)](x)$

$1 = g(x) - 3[f(x) - g(x)(x^3 + 5x^2 + 5x + 3)](x)$

$1 = g(x)[-3f(x) +3g(x)(x^3 + 5x^2 + 5x + 3)](x)$

$1 = g(x)+3g(x)(x^4 + 5x^3 + 5x^2 + 3x)-f(x)(3x)$

$1 = g(x)+g(x)(3x^4 + x^3 + x^2 + 2x)-f(x)(3x)$

$1 = g(x)(3x^4 + x^3 + x^2 + 2x+1)-f(x)(3x)$

\end{question}

\begin{question}{1c}$f(x) = x^4 + x^3 - x^2 - 2x - 2$ and $g(x) = x^3 - 1$ in $\Q[x]$
Euclidean Algorithm:

$(x^4 + x^3 - x^2 - 2x - 2) = (x^3 - 1)(x + 1) + (-x^2 - x - 1)$ %pulled - so x^2 + x + 1

$(x^3 - 1) = (x^2 + x + 1)(x - 1) + (0)$

Back Substitution:

$(x^2 + x + 1) = -1[f(x) - g(x)(x + 1)]$

$(x^2 + x + 1) = g(x)(x + 1)-f(x)$

\end{question}

\end{question}
% --------------------------------------------------------------
%                         Q2
% --------------------------------------------------------------
\begin{question}{2}
Which of the following is NOT an immediate consequence of the Fundamental Theorem
of Algebra?

\begin{question}{2a} Every $f(x) \in \R[x] \backslash \R$ has a root in $\C$.
\end{question}
\begin{question}{2b} Every irreducible polynomial $f(x) \in \R[x] has degree$ 1 or 2.
\end{question}
\begin{question}{2c} The only irreducible polynomials in $\C[x]$ are those of degree 1.
\end{question}
\begin{question}{2d} Everymonic polynomial $f(x) \in \R[x] \backslash \R$ will factor as a product of polynomials of degree 1 in $\C[x]$.
\end{question}

\end{question}
% --------------------------------------------------------------
%                         Q3
% --------------------------------------------------------------
\begin{question}{3}
Find the unique factorization of $f(x)$ as a product of monic irreducible polynomials
in the indicated polynomial ring $F[x]$. Explain why each of your polynomial factors is
irreducible in $F[x]$.

\begin{question}{3a} $f(x) = x^3 - 5x^2 + 9x - 45$ in $\C[x]$.
\end{question}
\begin{question}{3b} $f(x) = x^3 + x - 10$ in $\R[x]$.
\end{question}
\begin{question}{3c} $f(x) = x^4 + x^3 - 3x^2 + 2x + 5$ in $\Q[x]$.
\end{question}
\begin{question}{3d} $f(x) = x^4 + x^3 + 2x^2 + 2$ in $\Z_3[x]$.
\end{question}
\begin{question}{3e} $f(x) = x^4 + x^3 + x^2 + x$ in $\Z_2[x]$.
\end{question}
\begin{question}{3f} $f(x) = x^3 + 2x^2 + 3x + 4$ in $\Z_5[x]$.
\end{question}

\end{question}
% --------------------------------------------------------------
%                         Q4
% --------------------------------------------------------------
\begin{question}{4}
Let $D$ be an integral domain. Let $Q(D)$ be the field of fractions of $D$; i.e. the set of
equivalence classes of the set of all $D$-fractions under the equivalence of fractions relation.
Prove that addition of fractions is a well-defined binary operation on $Q(D)$: that is, prove

$\frac{a_1}{b_1} =_F \frac{a_2}{b_2}$ and 
$\frac{c_1}{d_1} =_F \frac{c_2}{d_2} \implies$
$\frac{a_1d_1 + c_1b_1}{b_1d_1} =_F \frac{a_2d_2 + c_2b_2}{b_2d_2}$

\end{question}
% --------------------------------------------------------------
%                         Q5
% --------------------------------------------------------------

\begin{question}{5}
 Find the partial fraction decomposition of these rational polynomials in the given field
of rational polynomials.
\end{question}

\begin{question}{5a} $\frac{1}{x^2+x}$ in $\Z_2[x]$
\end{question}
\begin{question}{5b} $\frac{7x^2+x-3}{x^3-x^2}$ in $\Q[x]$
\end{question}
\begin{question}{5c} $\frac{4x+2}{x^3+2x^2+4x+3}$ in $\Z_5[x]$
\end{question}


\end{document}
