%\documentclass{article}
%\usepackage[utf8]{inputenc}

%\title{a}
%\author{Quinn Neumiiller}
%\date{September 2013}

%\begin{document}

%\maketitle

%\section{Introduction}

%\end{document}

% --------------------------------------------------------------
% This is all preamble stuff that you don't have to worry about.
% Head down to where it says "Start here"
% --------------------------------------------------------------
 
\documentclass[12pt]{article}
 
\usepackage[margin=1in]{geometry} 
\usepackage{amsmath,amsthm,amssymb}
\usepackage{bbm}
 
\newcommand{\N}{\mathbb{N}}
\newcommand{\Z}{\mathbb{Z}}
\newcommand{\R}{\mathbb{R}}
\newcommand{\Q}{\mathbb{Q}}

 
\newenvironment{theorem}[2][Theorem]{\begin{trivlist}
\item[\hskip \labelsep {\bfseries #1}\hskip \labelsep {\bfseries #2.}]}{\end{trivlist}}
\newenvironment{lemma}[2][Lemma]{\begin{trivlist}
\item[\hskip \labelsep {\bfseries #1}\hskip \labelsep {\bfseries #2.}]}{\end{trivlist}}
\newenvironment{exercise}[2][Exercise]{\begin{trivlist}
\item[\hskip \labelsep {\bfseries #1}\hskip \labelsep {\bfseries #2.}]}{\end{trivlist}}
\newenvironment{problem}[2][Problem]{\begin{trivlist}
\item[\hskip \labelsep {\bfseries #1}\hskip \labelsep {\bfseries #2.}]}{\end{trivlist}}
\newenvironment{question}[2][Question]{\begin{trivlist}
\item[\hskip \labelsep {\bfseries #1}\hskip \labelsep {\bfseries #2.}]}{\end{trivlist}}
\newenvironment{corollary}[2][Corollary]{\begin{trivlist}
\item[\hskip \labelsep {\bfseries #1}\hskip \labelsep {\bfseries #2.}]}{\end{trivlist}}
 
\begin{document}

% --------------------------------------------------------------
%                         Samples
% --------------------------------------------------------------


%\begin{theorem}{x.yz} %You can use theorem, exercise, problem, or question here.  Modify x.yz to be whatever number you are proving
%Delete this text and write theorem statement here.
%\end{theorem}

%Blah, blah, blah. \mathbb{Z} Here is an example of the \texttt{align} environment:
%%Note 1: The * tells LaTeX not to number the lines.  If you remove the *, be sure to remove it below, too.
%%Note 2: Inside the align environment, you do not want to use $-signs.  The reason for this is that this is already a math environment. This is why we have to include \text{} around any text inside the align environment.
%\begin{align*}
%\sum_{i=1}^{k+1}i & = \left(\sum_{i=1}^{k}i\right) +(k+1)\\ 
%& = \frac{k(k+1)}{2}+k+1 & (\text{by inductive hypothesis})\\
%& = \frac{k(k+1)+2(k+1)}{2}\\
%& = \frac{(k+1)(k+2)}{2}\\
%& = \frac{(k+1)((k+1)+1)}{2}.
%\end{align*}
%\end{proof}
% 
%\begin{theorem}{x.yz}
%Let $n\in \Z$.  Then yada yada.
%\end{theorem}
% 
%\begin{proof}
%Blah, blah, blah.  I'm so smart.
%\end{proof}

% --------------------------------------------------------------
%                         Start here
% --------------------------------------------------------------
 
\title{Math 223 \\
Assignment 2 \underline{Episode 2}: The Algebraic Systems, Groups, Rings, and Fields Strikes Back
}
\author{Quinn Neumiiller} %if necessary, replace with your course title
 
\maketitle

% --------------------------------------------------------------
%                         Q2
% --------------------------------------------------------------

\begin{question}{2b}
Give the operation table for a binary operation on $A=\{a,b,c,d\}$ that has an identity and has inverses, but is not commutative. Show that your operation is not associative.
\end{question}

\begin{tabular}{l|llll}
$\star$ & a & b & c & d \\
\hline
a       & a & b & c & d \\
b       & b & a & c & d \\
c       & c & d & a & b \\
d       & d & c & b & a
\end{tabular}

$(c \star b) \star d =  c \star (b \star d)$

$d \star d =  c \star d$

$a\neq b$



% --------------------------------------------------------------
%                         Q3
% --------------------------------------------------------------

\begin{question}{3b}
$\Z^+$ under multiplication.
\end{question}

\underline{Inverse?} \textbf{False.}

$\forall z,y \in \Z^+$ : $z * y = e = y * z$

$y=6$

$z * y = 1 = y * z$

$z * 6 = 1 = 6 * z$

$z * 6 = 1 $

$z = \frac{1}{6} $

$\frac{1}{6} \notin \Z^+$. This holds for all values of $z,y$ except, in the case that $y=1$, $z= \frac{1}{1} = 1, 1 \in \Z^+$



\begin{question}{3e}
the set of even integers with respect to multiplication
\end{question}

\underline{Identity?} \textbf{False. $1 \notin \{$Even Integers$\}$}



\begin{question}{3f}
the set of positive divisors of 10 with respect to GCD.
\end{question}

\underline{Inverse?} \textbf{False.}

Given that the gcd-identity is 10.

$a=10$ $b=5$

$gcd(10,5) = 5$ 

$5 \neq 10$

\begin{question}{3g}
the set of surjective functions from $A$ to $A$ with respect to composition.
\end{question}

\underline{Inverse?} \textbf{False. There is a right inverse, but not a left inverse.}

$f(x) \circ g(x) = e(x) = g(x) \circ f(x)$

It is possible given $g(x) = y$ that $f(g(x)) = f(y) = y$ but impossible for $y(f(x)) = e(x)$

$f(x) \circ g(x) = e(x)$

$f(g(x)) = e(x)$

$f(y) = e(x)$

\begin{question}{3i}
the set of polynomials with real coefficients having constant term 1 under multiplication.
\end{question}

\textbf{yes it is a group}

% --------------------------------------------------------------
%                         Q4
% --------------------------------------------------------------

\begin{question}{4}
Decide if the following sets with two operations are rings. If it is not, find the first property of the ring definition that fails to hold. If it is a ring, also determine if it is an integral domain or a field.
\end{question}

\begin{question}{4d}
The set of functions $\R$ to $\R$ under + of values and function composition $\circ$
\end{question}

\underline{Checking for field, Inverse?} \textbf{False. The set of composition functions includes more than bijections. Bijections are the only composition functions to satisfy inverses. ($\R$, +, $\circ$) is a ring, and a integral domain.}

% --------------------------------------------------------------
%                         End Document
% --------------------------------------------------------------
\end{document}