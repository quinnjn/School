\documentclass{article}
%--Packages------------
\usepackage{fancyvrb}
\usepackage{alltt}

%----------------------

%--Variables-----------
\newcommand{\AssignmentNum}{2}
\newcommand{\QuestionNum}{1}
\newcommand{\AssignmentTitle}{Are you satised with Constraint Satisfaction?}
\newcommand{\AssignmentDate}{\today}
%----------------------

%--Title & Author------
\title{CMPT 317, Term 2, 2013\\
Assignment \AssignmentNum\\
\AssignmentTitle
}
\author{
	\begin{tabular}{ l r }
	  Name: & Quinn Neumiiller \\
	  NSID: & qjn162 \\
	  Student \# & 11065618 \\
	\end{tabular}
}
\date{\AssignmentDate}
%----------------------

%--Document------------
\begin{document}
   \maketitle
   
   \section{Compiling}
   Not required for Python

   \section{Execution}
   To execute, change the terminal directory to where this PDF file sits.
   Then in the terminal execute:
	\begin{alltt}
	python src/Q\QuestionNum.py
	\end{alltt}

  % \section{Usage}
  %   \subsection{Simple Interface}
  %     Options in the Simple Interface are as followed:
  %     \begin{itemize}
  %       \item Q or q - Quits the program
  %       \item C or c or [Enter Key] - Allows the board to progress to the next generation
  %       \item L or l - Loads a new board, you are then required to enter the relative path towards the new board to load.
  %     \end{itemize}
  %   \subsection{Command Line Arguments}
  %     To make it simpler on loading the board, the first command line argument can be the board you wish to load.

  %     Example:
  %     \begin{alltt}
		% python src/Q\QuestionNum.py tests/A1_Sample1.text
	 %  \end{alltt}
	 %  This is designed to make it easier by using the terminals auto complete option for relative files.


\end{document}
%----------------------

