%\documentclass{article}
%\usepackage[utf8]{inputenc}

%\title{a}
%\author{Quinn Neumiiller}
%\date{September 2013}

%\begin{document}

%\maketitle

%\section{Introduction}

%\end{document}

% --------------------------------------------------------------
% This is all preamble stuff that you don't have to worry about.
% Head down to where it says "Start here"
% --------------------------------------------------------------
 
\documentclass[12pt]{article}
 
\usepackage[margin=1in]{geometry} 
\usepackage{amsmath,amsthm,amssymb}
\usepackage{bbm}


\usepackage{fancyhdr}
\usepackage[us,12hr]{datetime} % `us' makes \today behave as usual in TeX/LaTeX
\fancypagestyle{plain}{
\fancyhf{}
\rfoot{Compiled on {\ddmmyyyydate\today} at \currenttime}
\lfoot{Page \thepage}
\renewcommand{\headrulewidth}{0pt}}
\pagestyle{plain}

 
\newcommand{\N}{\mathbb{N}}
\newcommand{\Z}{\mathbb{Z}}
\newcommand{\R}{\mathbb{R}}
\newcommand{\Q}{\mathbb{Q}}

 
\newenvironment{theorem}[2][Theorem]{\begin{trivlist}
\item[\hskip \labelsep {\bfseries #1}\hskip \labelsep {\bfseries #2.}]}{\end{trivlist}}
\newenvironment{lemma}[2][Lemma]{\begin{trivlist}
\item[\hskip \labelsep {\bfseries #1}\hskip \labelsep {\bfseries #2.}]}{\end{trivlist}}
\newenvironment{exercise}[2][Exercise]{\begin{trivlist}
\item[\hskip \labelsep {\bfseries #1}\hskip \labelsep {\bfseries #2.}]}{\end{trivlist}}
\newenvironment{problem}[2][Problem]{\begin{trivlist}
\item[\hskip \labelsep {\bfseries #1}\hskip \labelsep {\bfseries #2.}]}{\end{trivlist}}
\newenvironment{question}[2][Question]{\begin{trivlist}
\item[\hskip \labelsep {\bfseries #1}\hskip \labelsep {\bfseries #2.}]}{\end{trivlist}}
\newenvironment{corollary}[2][Corollary]{\begin{trivlist}
\item[\hskip \labelsep {\bfseries #1}\hskip \labelsep {\bfseries #2.}]}{\end{trivlist}}
 
\begin{document}

% --------------------------------------------------------------
%                         Samples
% --------------------------------------------------------------


%\begin{theorem}{x.yz} %You can use theorem, exercise, problem, or question here.  Modify x.yz to be whatever number you are proving
%Delete this text and write theorem statement here.
%\end{theorem}

%Blah, blah, blah. \mathbb{Z} Here is an example of the \texttt{align} environment:
%%Note 1: The * tells LaTeX not to number the lines.  If you remove the *, be sure to remove it below, too.
%%Note 2: Inside the align environment, you do not want to use $-signs.  The reason for this is that this is already a math environment. This is why we have to include \text{} around any text inside the align environment.
%\begin{align*}
%\sum_{i=1}^{k+1}i & = \left(\sum_{i=1}^{k}i\right) +(k+1)\\ 
%& = \frac{k(k+1)}{2}+k+1 & (\text{by inductive hypothesis})\\
%& = \frac{k(k+1)+2(k+1)}{2}\\
%& = \frac{(k+1)(k+2)}{2}\\
%& = \frac{(k+1)((k+1)+1)}{2}.
%\end{align*}
%\end{proof}
% 
%\begin{theorem}{x.yz}
%Let $n\in \Z$.  Then yada yada.
%\end{theorem}
% 
%\begin{proof}
%Blah, blah, blah.  I'm so smart.
%\end{proof}

% --------------------------------------------------------------
%                         Start here
% --------------------------------------------------------------
 
\title{Math 223 \\
Assignment 4: Applications of Modular Arithmetic to Codes and
Cryptography
}
\author{Quinn Neumiiller} %if necessary, replace with your course title
 
\maketitle

% --------------------------------------------------------------
%                         Q1
% --------------------------------------------------------------

\begin{question}{1}
Check-digit codes. Check whether or not the following codewords are valid for the
code given.
\end{question}

% --------------------------------------------------------------
%                         Q1a
% --------------------------------------------------------------

\begin{question}{1a}
the UPC number 7-80133-19831-7

%\begin{tabular}{l|l|l|l}
%i & a & ia & ia mod 10  \\
%\hline
%7 & 3 & 21 & 1 \\
%8 & 1 & 8  & 8 \\
%0 & 3 & 0  & 0 \\
%1 & 1 & 1  & 1 \\
%3 & 3 & 9  & 9 \\
%3 & 1 & 3  & 3 \\
%1 & 3 & 3  & 3 \\
%9 & 1 & 9  & 9 \\
%8 & 3 & 21 & 1 \\
%3 & 1 & 3  & 3 \\
%1 & 3 & 3  & 3 \\
%7 & 1 & 7  & 7
%\end{tabular}
\begin{tabular}{l|cccccccccccc}
\hline
i     & 7  & 8 & 0 & 1 & 3 & 3 & 1 & 9 & 8 & 3 & 1 & 7 \\
a     & 3  & 1 & 3 & 1 & 3 & 1 & 3 & 1 & 3 & 1 & 3 & 1\\
\hline
(i*a) & 21 & 8 & 0 & 1 & 9 & 3 & 3 & 9 & 24 & 3 & 3 & 7
\end{tabular}

\begin{align*}
=&21 + 8 + 0 + 1 + 9 + 3 + 3 + 9 + 24 + 3 + 3 + 7\textrm{ mod } 10\\
=&91 \textrm{ mod } 10\\
=&1 \textrm{ mod } 10\\
\end{align*}

Not Valid.
\end{question}

% --------------------------------------------------------------
%                         Q1b
% --------------------------------------------------------------

\begin{question}{1b}
the ISBN-10 number 0-87150-334-X.

\begin{tabular}{l|cccccccccc}
\hline
i     & 0  & 8  & 7  & 1  & 5  & 0  & 3  & 3  & 4 & 10 \\
a     & 10 & 9  & 8  & 7  & 6  & 5  & 4  & 3  & 2 & 1 \\
\hline
(i*a) & 0  & 72 & 56 & 7  & 30 & 0  & 12 & 9  & 8 & 10
\end{tabular}

\begin{align*}
=&0  + 72 + 56 + 7  + 30 + 0  + 12 + 9  + 8 + 10\textrm{ mod } 11\\
=&204 \textrm{ mod } 11\\
=&6 \textrm{ mod } 11
\end{align*}

Not Valid.
\end{question}

% --------------------------------------------------------------
%                         Q1c
% --------------------------------------------------------------

\begin{question}{1c}
the ISBN-10 number 0-13-319831-0.

\begin{tabular}{l|cccccccccc}
\hline
i     & 0  & 1  & 3  & 3  & 1  & 9  & 8  & 3  & 1 & 0 \\
a     & 10 & 9  & 8  & 7  & 6  & 5  & 4  & 3  & 2 & 1 \\
\hline
(i*a) & 0  & 9  & 24 & 21 & 6  & 45 & 32 & 9  & 2 & 0
\end{tabular}

\begin{align*}
=& 0  + 9  + 24 + 21 + 6  + 45 + 32 + 9  + 2 + 0 \textrm{ mod } 11\\
=& 148 \textrm{ mod } 11\\
=& 5 \textrm{ mod } 11
\end{align*}

Not Valid.
\end{question}

% --------------------------------------------------------------
%                         Q1d
% --------------------------------------------------------------

\begin{question}{1d}
the ISBN-13 number 978-1-4292-6009-1

\begin{tabular}{l|ccccccccccccc}
\hline
i     & 9  & 7  & 8  & 1  & 4  & 2  & 9  & 2  & 6  & 0 & 0 & 9 & 1\\
a     & 1  & 3  & 1  & 3  & 1  & 3  & 1  & 3  & 1  & 3 & 1 & 3 & 1 \\ 
\hline
(i*a) & 9  & 21 & 8  & 3  & 4  & 6  & 9  & 6  & 6  & 0 & 0 & 27 & 1
\end{tabular}

\begin{align*}
=& 9  + 21 + 8  + 3  + 4  + 6  + 9  + 6  + 6  + 0 + 0 + 27 + 1 \textrm{ mod } 10 \\
=& 100 \textrm{ mod } 10 \\
=& 0 \textrm{ mod } 10 \\
\end{align*}

Valid.
\end{question}

% --------------------------------------------------------------
%                         Q1e
% --------------------------------------------------------------

\begin{question}{1e}
the Bank ID number 145-79429-1.

\begin{tabular}{l|ccccccccc}
\hline
i     & 1  & 4  & 5  & 7  & 9  & 4  & 2  & 9  & 1 \\
a     & 7  & 3  & 9  & 7  & 3  & 9  & 7  & 3  & 9 \\ 
\hline
(i*a) & 7  & 12 & 45 & 49 & 27 & 36 & 14 & 27 & 9
\end{tabular}

\begin{align*}
=& 7  + 12 + 45 + 49 + 27 + 36 + 14 + 27 + 9 \textrm{ mod } 10\\
=& 226 \textrm{ mod } 10\\
=& 6 \textrm{ mod } 10\\
\end{align*}

Not valid, $5 \neq 6$
\end{question}

% --------------------------------------------------------------
%                         Q2
% --------------------------------------------------------------

\begin{question}{2}
Error correction? You, the bookseller, have entered the following ISBN-13 number for
the book you are trying to ring through the till:
978-0-321-75277-2
Although the codeword appears to be valid in the ISBN-13 system, it does not correspond
to a book in your store's inventory. Assuming the ISBN is incorrect only because of a switch
of 2 adjacent digits, what is its correct ISBN-13?
(For fun: look up the actual title, author, and publisher of this book online.)

\begin{align*}
7-9 =&-2\\
8-7 =& 1\\
0-8 =& -8\\
3-0=&3\\
2-3=&-1\\
1-2=&-1\\
7-1=&6\\
5-7=&-2\\
2-5=&-3\\
2-7=&-5
\end{align*}
Try switching 2 and 7.

\begin{tabular}{l|ccccccccccccc}
\hline
i     & 9  & 7  & 8  & 0  & 3  & 2  & 1  & 7  & 5  & 7 & 2 & 7 & 2\\
a     & 1  & 3  & 1  & 3  & 1  & 3  & 1  & 3  & 1  & 3 & 1 & 3 & 1 \\ 
\hline
(i*a) & 9  & 21 & 8  & 0  & 3  & 6  & 1  & 21 & 5  & 21& 2 &21 & 2
\end{tabular}

\begin{align*}
=& 9  + 21 + 8  + 0  + 3  + 6  + 1  + 21 + 5  + 21+ 2 +21 + 2 \textrm{ mod } 10\\
=& 120 \textrm{ mod } 10\\
=& 0 \textrm{ mod } 10
\end{align*}

ISBN-13: 978-0-321-75727-2

Title: Statistics : Informed Decisions Using Data with CD 4th

Author: Michael Sullivan III

Publisher: Addison Wesley

\end{question}

% --------------------------------------------------------------
%                         Q3
% --------------------------------------------------------------

\begin{question}{3}
Passport numbers. The identication code used for international passports is:
[(6 or 9-digit passport number)-(check digit)]-
(3 letter country code)-[(6-digit birth date)-(check digit)]-
(an M or F)-[(6-digit expiry date)-(check digit)]$>>>>>$(overall check digit)
Each check digit is calculated using the check vector pattern (7; 3; 1; 7; 3; 1; :::) mod 10.
The overall check digit is calculated with all of the preceding digits in the passport ID,
including check digits, excluding letters.
\end{question}

% --------------------------------------------------------------
%                         Q3a
% --------------------------------------------------------------

\begin{question}{3a}
Verify the validity of the passport number
044455533-1-USA-460920-5-M-040913-1$>>>>>>>>>>>>$8

For my purposeses, I'm going to drop all the letters, and seperate the sections.
[[044455533-1]-[460920-5]-[040913-1]$>>>>>>>>>>>>$8]

Checking passport number.

\begin{tabular}{l|ccccccccccccc}
\hline
i     & 0  & 4  & 4  & 4  & 5  & 5  & 5  & 3  & 3  & 1 \\
a     & 7  & 3  & 1  & 7  & 3  & 1  & 7  & 3  & 1  & 7 \\
\hline
(i*a) & 0  & 12 & 4  & 28 & 15 & 5  & 35 & 9  & 3  & 7
\end{tabular}

\begin{align*}
=& 0  + 12 + 4  + 28 + 15 + 5  + 35 + 9  + 3  + 7 \textrm{ mod } 10\\
=& 118 \textrm{ mod } 10\\
=& 8 \textrm{ mod } 10
\end{align*}



Checking birth date

\begin{tabular}{l|ccccccccccccc}
\hline
i     & 4  & 6  & 0  & 9  & 2  & 0  & 5 \\
a     & 7  & 3  & 1  & 7  & 3  & 1  & 7 \\
\hline
(i*a) & 28 & 18 & 0  & 63 & 6  & 0  & 35
\end{tabular}

\begin{align*}
=& 28 + 18 + 0  + 63 + 6  + 0  + 35 \textrm{ mod } 10\\
=& 150 \textrm{ mod } 10\\
=& 0 \textrm{ mod } 10
\end{align*}

Birth date is valid.



Checking Expiry date

\begin{tabular}{l|ccccccccccccc}
\hline
i     & 0  & 4  & 0  & 9  & 1  & 3  & 1 \\
a     & 7  & 3  & 1  & 7  & 3  & 1  & 7 \\
\hline
(i*a) & 0  & 12 & 0  & 63 & 3  & 3  & 7 
\end{tabular}

\begin{align*}
=& 0  + 12 + 0  + 63 + 3  + 3  + 7  \textrm{ mod } 10\\
=& 88 \textrm{ mod } 10\\
=& 8 \textrm{ mod } 10
\end{align*}

Expiry date is invalid.




% --------------------------------------------------------------
%                         Q3b
% --------------------------------------------------------------

\begin{question}{3b}
Determine check digits to complete the following (fake) Canadian passport number:
203241-?-CAN-840712-?-F-090215-?$>>>>>>>>>>$?

Finding passport number.

\begin{tabular}{l|ccccccccccccc}
\hline
i     & 2 & 0 & 3 & 2 & 4 & 1\\
a     & 7 & 3 & 1 & 7 & 3 & 1\\
\hline
(i*a) & 14& 0 & 3 & 14& 12& 1
\end{tabular}

\begin{align*}
=& 14+ 0 + 3 + 14+ 12+ 1 \textrm{ mod } 10\\
=& 44\textrm{ mod } 10\\
=& 4 \textrm{ mod } 10\\
\end{align*}

\begin{align*}
=& 4 + 7(x)\textrm{ mod } 10\\
11 =& 4 + 7(1)\textrm{ mod } 10\\
18 =& 4 + 7(2)\textrm{ mod } 10\\
25 =& 4 + 7(3)\textrm{ mod } 10\\
32 =& 4 + 7(4)\textrm{ mod } 10\\
39 =& 4 + 7(5)\textrm{ mod } 10\\
46 =& 4 + 7(6)\textrm{ mod } 10\\
53 =& 4 + 7(7)\textrm{ mod } 10\\
60 =& 4 + 7(8)\textrm{ mod } 10\\
\end{align*}

Check digit for the passport number is 8.



Checking birth date

\begin{tabular}{l|ccccccccccccc}
\hline
i     & 8  & 4  & 0  & 7  & 1  & 2 \\
a     & 7  & 3  & 1  & 7  & 3  & 1 \\
\hline
(i*a) & 56 & 12 & 0  & 49 & 3  & 2
\end{tabular}

\begin{align*}
=& 56 + 12 + 0 + 49 + 3 + 2 \textrm{ mod } 10\\
=& 122 \textrm{ mod } 10\\
=& 2 \textrm{ mod } 10
\end{align*}

\begin{align*}
=& 2 + 7(x)\textrm{ mod } 10\\
9=& 2 + 7(1)\textrm{ mod } 10\\
16=& 2 + 7(2)\textrm{ mod } 10\\
23=& 2 + 7(3)\textrm{ mod } 10\\
30=& 2 + 7(4)\textrm{ mod } 10\\
\end{align*}

\begin{align*}
=& 56 + 12 + 0 + 49 + 3 + 2 + 7(4)\textrm{ mod } 10\\
=& 56 + 12 + 0 + 49 + 3 + 2 + 28\textrm{ mod } 10\\
=& 150 \textrm{ mod } 10\\
=& 0 \textrm{ mod } 10
\end{align*}

Check digit for the birth date is 4.



Checking Expiry date
090215

\begin{tabular}{l|ccccccccccccc}
\hline
i     & 0  & 9  & 0  & 2  & 1  & 5  \\
a     & 7  & 3  & 1  & 7  & 3  & 1  \\
\hline
(i*a) & 0  & 27 & 0  & 14 & 3  & 5
\end{tabular}

\begin{align*}
=& 0  + 27 + 0  + 14 + 3  + 5  \textrm{ mod } 10\\
=& 49 \textrm{ mod } 10\\
=& 9 \textrm{ mod } 10
\end{align*}

\begin{align*}
=& 9 + 7(x)\textrm{ mod } 10\\
16=& 9 + 7(1)\textrm{ mod } 10\\
23=& 9 + 7(2)\textrm{ mod } 10\\
30=& 9 + 7(3)\textrm{ mod } 10\\
\end{align*}

\begin{align*}
=& 0  + 27 + 0  + 14 + 3  + 5 +7(3) \textrm{ mod } 10\\
=& 0  + 27 + 0  + 14 + 3  + 5 + 21 \textrm{ mod } 10\\
=& 70 \textrm{ mod } 10\\
=& 0 \textrm{ mod } 10
\end{align*}

Check digit for expiry date is 3.



Checking the full passport, aka: 203241884071240902153?

\begin{tabular}{l|ccccccccccccccccccccc}
\hline
i     &  2  & 0  & 3  & 2  & 4  & 1  & 8  & 8  & 4  & 0  & 7  & 1  & 2  & 4  & 0  & 9  & 0  & 2  & 1  & 5  & 3 \\
a     & 7   & 3  & 1  & 7  & 3  & 1  & 7  & 3  & 1  & 7  & 3  & 1  & 7  & 3  & 1  & 7  & 3  & 1  & 7  & 3  & 1 \\
\hline
(i*a) & 14 & 0 & 3  & 14  & 12 & 1  & 56 & 24 & 4  & 0  & 21 & 1  & 14 & 12 & 0  & 63 & 0  & 2  & 7  & 15 & 3
\end{tabular}

\begin{align*}
=& 14  + 3   + 14   + 12  + 1   + 56  + 24  + 4   + 21  + 1   + 14  + 12  + 63  + 2   + 7   + 15  + 3 \textrm{ mod } 10\\
=& 266 \textrm{ mod } 10\\
=& 6 \textrm{ mod } 10\\
\end{align*}

\begin{align*}
=& 6 + 1(x)\textrm{ mod } 10\\
=& 6 + x\textrm{ mod } 10\\
10=& 6 + 4\textrm{ mod } 10\\
\end{align*}

\begin{align*}
=& 14  + 3   + 14   + 12  + 1   + 56  + 24  + 4   + 21  + 1   + 14  + 12  + 63  + 2   + 7   + 15  + 3 +1(4) \textrm{ mod } 10\\
=& 14  + 3   + 14   + 12  + 1   + 56  + 24  + 4   + 21  + 1   + 14  + 12  + 63  + 2   + 7   + 15  + 3 +4    \textrm{ mod } 10\\
=& 270 \textrm{ mod } 10\\
=& 0 \textrm{ mod } 10
\end{align*}

Final checkdigit on the passport is 4.

Total valid passport is: 203241-8-CAN-840712-4-F-090215-3$>>>>>>>>>>$4


\end{question}

% --------------------------------------------------------------
%                         Q4
% --------------------------------------------------------------

\begin{question}{4}
Hacking RSA cryptosystems.
\end{question}

% --------------------------------------------------------------
%                         Q4a
% --------------------------------------------------------------

\begin{question}{4a}
Suppose an RSA cryptosystem has public key ($n$, $e$) = (6282, 197). Find the associated private key ($\phi(n)$, $d$)
\end{question}

% --------------------------------------------------------------
%                         Q4b
% --------------------------------------------------------------

\begin{question}{4b}
Let ($n$, $e$) = ($9991$, $11$) be the public key for an RSA cryptosystem that encrypts
letters using the standard ASCII system. Decipher the transmitted message: 5752 7155
\end{question}

% --------------------------------------------------------------
%                         Q5
% --------------------------------------------------------------

\begin{question}{5}
Digital signatures. Bob's RSA cryptosystem uses public key (9379, 1837). Alice's uses
public key (8453, 7).
\end{question}

% --------------------------------------------------------------
%                         Q5a
% --------------------------------------------------------------

\begin{question}{5a}
Verify that Bob's private key is (9184, 5) and Alice's private key is (8268, 7087).
\end{question}

% --------------------------------------------------------------
%                         Q5b
% --------------------------------------------------------------

\begin{question}{5b}
Alice wants to send the signed message $ALGEBRA$ to Bob. She will encrypt in
2-letter (4-digit) blocks using standard ASCII. What message should she transmit? Provide
the calculations Bob will use to verify the signed message he receives without knowledge of
Alice's private key.
\end{question}

\end{document}