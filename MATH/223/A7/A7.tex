%\documentclass{article}
%\usepackage[utf8]{inputenc}

%\title{a}
%\author{Quinn Neumiiller}
%\date{September 2013}

%\begin{document}

%\maketitle

%\section{Introduction}

%\end{document}

% --------------------------------------------------------------
% This is all preamble stuff that you don't have to worry about.
% Head down to where it says "Start here"
% --------------------------------------------------------------
 
\documentclass[12pt]{article}
 
\usepackage[margin=1in]{geometry} 
\usepackage{amsmath,amsthm,amssymb}
%\usepackage{bbm}

\usepackage{polynom}

\usepackage{fancyhdr}
\usepackage[us,12hr]{datetime} % `us' makes \today behave as usual in TeX/LaTeX
\fancypagestyle{plain}{
\fancyhf{}
\rfoot{Compiled on {\ddmmyyyydate\today} at \currenttime}
\lfoot{Page \thepage}
\renewcommand{\headrulewidth}{0pt}}
\pagestyle{plain}

 
\newcommand{\N}{\mathbb{N}}
\newcommand{\Z}{\mathbb{Z}}
\newcommand{\R}{\mathbb{R}}
\newcommand{\Q}{\mathbb{Q}}
\newcommand{\C}{\mathbb{C}}

\newcommand{\z}{\overline{z}}
 
\newenvironment{theorem}[2][Theorem]{\begin{trivlist}
\item[\hskip \labelsep {\bfseries #1}\hskip \labelsep {\bfseries #2.}]}{\end{trivlist}}
\newenvironment{lemma}[2][Lemma]{\begin{trivlist}
\item[\hskip \labelsep {\bfseries #1}\hskip \labelsep {\bfseries #2.}]}{\end{trivlist}}
\newenvironment{exercise}[2][Exercise]{\begin{trivlist}
\item[\hskip \labelsep {\bfseries #1}\hskip \labelsep {\bfseries #2.}]}{\end{trivlist}}
\newenvironment{problem}[2][Problem]{\begin{trivlist}
\item[\hskip \labelsep {\bfseries #1}\hskip \labelsep {\bfseries #2.}]}{\end{trivlist}}
\newenvironment{question}[2][Question]{\begin{trivlist}
\item[\hskip \labelsep {\bfseries #1}\hskip \labelsep {\bfseries #2.}]}{\end{trivlist}}
\newenvironment{corollary}[2][Corollary]{\begin{trivlist}
\item[\hskip \labelsep {\bfseries #1}\hskip \labelsep {\bfseries #2.}]}{\end{trivlist}}
 
\begin{document}

% --------------------------------------------------------------
%                         Samples
% --------------------------------------------------------------
 
\title{Math 223 \\
Assignment 7: Examples of Groups
}
\author{Quinn Neumiiller}
 
\maketitle

% --------------------------------------------------------------
%                         Q1
% --------------------------------------------------------------

\begin{question}{Q1}
Perform the following calculations in $\C$:
\end{question}

\begin{question}{1a}
\begin{align*}
(2-3i)+(-1 +4i) &= (2-3i)+(-1 +4i)\\
&= 2 -3i -1 + 4i\\
&= 1 + i
\end{align*}
\end{question}

\begin{question}{1b}
\begin{align*}
(2-11i)(-4+3i) 
&= -8 + 6i + 44i - 33i^2\\ 
&= 50i + 33 -8 \\
&= 50i +25 
\end{align*}
\end{question}
\begin{question}{1c}
\begin{align*}
(4+3i)^{-1}
	&= \frac{4-3i}{4^2+3^3} \\
	&= \frac{4-3i}{16+9} \\
	&= \frac{4-3i}{25} \\
\end{align*}
\end{question}
\begin{question}{1d}
\begin{align*}
(\sqrt{2}i)^{8}
 &= (\sqrt{2})^8(i)^8\\ 
 &= ((\sqrt{2})^2)^4((i)^2)^4\\ 
 &= (2)^4(-1)^4\\ 
 &= 16
\end{align*}
\end{question}
\begin{question}{1e}
\begin{align*}
(2+3i)\overline{(3+4i)}
 &= (2+3i)(3-4i)\\ 
 &= 6-8i+9i-12i^2\\ 
 &= 6+i+12\\ 
 &= 18+i\\ 
\end{align*}
\end{question}
\begin{question}{1f}
Let $x=-2$ and $y=2\sqrt{3}$

\begin{align*}
r 
 &= \sqrt{x^2 + y^2} \\
 &= \sqrt{(-2)^2 + (2\sqrt{3})^2} \\
 &= \sqrt{4 + (4*3)} \\
 &= \sqrt{4 + 12} \\
 &= \sqrt{16} \\
 &= 4 \\
\end{align*}

\begin{align*}
\theta 
&= tan^{-1}(\frac{2\sqrt{3}}{-2}) \\
&= tan^{-1}(\sqrt{-3}) \\
&= -\frac{\pi}{3}
\end{align*}

\begin{align*}
(2\sqrt{3}i-2)^{6}
 &= (re^{i\theta})^6\\ 
 &= ((4)e^{i(-\frac{\pi}{3})})^6\\ 
 &= (4^6)e^{6i(-\frac{\pi}{3})}\\ 
 &= 4096e^{-2\pi i}\\ 
 &= 4096(\cos(-2\pi) + i\sin(-2\pi))\\ 
 &= 4096(1 + i0)\\ 
 &= 4096 
\end{align*}


\end{question}
\begin{question}{1g}
Let $x=1$ and $y=-1$
\begin{align*}
r
&= \sqrt{(1)^2+(-1)^2} \\
&= \sqrt{1+1} \\
&= \sqrt{2}
\end{align*}

\begin{align*}
\theta
 &= tan^{-1}(-\sqrt{3}) \\
 &= tan^{-1}(-\sqrt{3}) \\
 &= -\frac{\pi}{3}
\end{align*}

\begin{align*}
(1-i)^{4}
 &= (re^{i\theta})^4\\ 
 &= ((\sqrt{2})e^{i(-\frac{\pi}{4})})^4\\ 
 &= (\sqrt{2})^4e^{4i(-\frac{\pi}{4})}\\ 
 &= 4e^{-4i\pi}\\ 
 &= 4(\cos(-\pi)+i\sin(-\pi))\\
 &= 4(-1+i0)\\
 &= -4
\end{align*}
\end{question}

\begin{question}{1h} Express in polar form
Let $x=1$ and $y=-\sqrt{3}$

\begin{align*}
\theta
 &= tan^{-1}(-\sqrt{3}) \\
 &= tan^{-1}(-\sqrt{3}) \\
 &= -\frac{\pi}{3}
\end{align*}

\begin{align*}
r 
&= \sqrt{1^2 + (\sqrt{3})^2} \\
&= \sqrt{1 + 3} \\
&= \sqrt{4} \\
&= 2 \\
\end{align*}

\begin{align*}
(\frac{1}{4\sqrt{2}} - \frac{\sqrt{3i}}{4\sqrt{2}})^{9}
 &= (\frac{1}{4\sqrt{2}}(1-\sqrt{3i}))^{9} \\
 &= (\frac{1}{4\sqrt{2}}(re^{i\theta}))^{9} \\
 &= (\frac{1}{4\sqrt{2}}(2)e^{i(-\frac{\pi}{3}})))^{9} \\
 &= (\frac{1}{2\sqrt{2}}e^{i(-\frac{\pi}{3}})))^{9} \\
 &= (\frac{1}{2\sqrt{2}})^9   e^{-3\pi i} \\
 &= \frac{1}{8192\sqrt{2}}   e^{-3\pi i} \\
 &= \frac{1}{8192\sqrt{2}}   (\cos(-3\pi) +i\sin(-3\pi)) \\
 &= \frac{1}{8192\sqrt{2}}   (-1 +i0) \\
 &= -\frac{1}{8192\sqrt{2}}
\end{align*}
\end{question}


\begin{question}{1i}
\begin{align*}
(re^{i\theta})^{-1}
 &= \frac{e^{-i\theta}}{r}
\end{align*} 
\end{question}

\begin{question}{1j} Express in standard $x+yi$ form
\begin{align*}
(\sqrt{5}e^{\frac{2\pi i}{5}})(2\sqrt{5}e^{\frac{3\pi i}{5}})
 &= 2(\sqrt{5})^2e^{\pi i}\\ 
 &= 10(-1)\\ 
 &= -10 
\end{align*}
\end{question}

% --------------------------------------------------------------
%                         Q2
% --------------------------------------------------------------
\begin{question}{Q2}
Solve the following equations. Find all solutions for $z$ in $\C$. Express your answer in $x+yi$ form when convenient, otherwise leave them in polar form.
\end{question}

%z^n = r^n e^{in\theta}
%
%so if z^n = 5 ...
%r^n = 5
%thus, r=\sqrt[n]{5}
%
\begin{question}{2a}
$z^2=5$

$r = \sqrt{5}$

$\theta = 0$

$2\theta \in [0, 4\pi)$

$2\theta \in \{0, 2\pi\}$

$\theta \in \{0, \pi\}$

$z = \{\sqrt{5}e^{i{0}}, \sqrt{5}e^{i\pi}\}$

$z = \{\sqrt{5}, \sqrt{5}(\cos{\pi}+i\sin{\pi})\}$

$z = \{\sqrt{5}, \sqrt{5}(-1+i0\}$

$z = \{\sqrt{5}, -\sqrt{5}\}$


\end{question}
\begin{question}{2b}
$z^2 +2z +3 =0$

Apply the quadratic equation

\begin{align*}
\frac{-2\pm \sqrt{4-4(3)}}{2} 
	&= -1\pm \frac{\sqrt{-8}}{2} \\
	&= -1\pm \frac{2i\sqrt{2}}{2} \\
	&= -1\pm i\sqrt{2} 
\end{align*}
\end{question}

\begin{question}{2c}
$z^3=-1$

$r = \sqrt[3]{-1}$

$\theta = 0$

$3\theta \in [0, 6\pi)$

$3\theta \in \{0, 2\pi, 4\pi\}$

$\theta \in \{0, \frac{2\pi}{3}, \frac{4\pi}{3}\}$

$z = \{\sqrt[3]{-1}e^{i0}, \sqrt[3]{-1}e^{i\frac{2\pi}{3}}, \sqrt[3]{-1}e^{i\frac{4\pi}{3}}\}$ 

$z = \{\sqrt[3]{-1}(\cos(0)+i\sin(0)), \sqrt[3]{-1}e^{i\frac{2\pi}{3}}, \sqrt[3]{-1}e^{i\frac{4\pi}{3}}\}$ 

$z = \{\sqrt[3]{-1}, \sqrt[3]{-1}e^{i\frac{2\pi}{3}}, \sqrt[3]{-1}e^{i\frac{4\pi}{3}}\}$ 

\end{question}
\begin{question}{2d}
$z^3 + 6z^2 + 12z + 3 = 0$ (Hint: $(z+2)^3 = z^3 + 6z^2 + 12z + 8$)

$z^3 + 6z^2 + 12z + 8 = 5$ 

$(z+2)^3 = 5$ 

Let $\z = z+2$

$\z^3 = 5$

$r = \sqrt[3]{5}$

$\theta = 0$

$3\theta \in [0, 6\pi)$

$3\theta \in \{0, 2\pi, 4\pi\}$

$\theta \in \{0, \frac{2\pi}{3}, \frac{4\pi}{3}\}$

$\z = \{\sqrt[3]{5}e^{i0}, \sqrt[3]{5}e^{i\frac{2\pi}{3}}, \sqrt[3]{5}e^{i\frac{4\pi}{3}}\}$ 

$\z = \{\sqrt[3]{5}, \sqrt[3]{5}e^{i\frac{2\pi}{3}}, \sqrt[3]{5}e^{i\frac{4\pi}{3}}\}$ 

$(z+2) = \{\sqrt[3]{5}, \sqrt[3]{5}e^{i\frac{2\pi}{3}}, \sqrt[3]{5}e^{i\frac{4\pi}{3}}\}$ 

$z = \{\sqrt[3]{5}-2, \sqrt[3]{5}e^{i\frac{2\pi}{3}}-2, \sqrt[3]{5}e^{i\frac{4\pi}{3}}-2\}$ 

\end{question}
\begin{question}{2e}
$z^4 = 16 - 16i$

$\theta = tan^{-1}(-1) = -\frac{\pi}{4}$

$4\theta = [0, 8\pi)$

$4\theta = \{-\frac{\pi}{4}, 2\pi-\frac{\pi}{4}, 4\pi -\frac{\pi}{4}, 6\pi -\frac{\pi}{4}\}$

$4\theta = \{-\frac{\pi}{4}, \frac{7\pi}{4}, \frac{15\pi}{4}, \frac{23\pi}{4}\}$

$\theta = \{-\frac{\pi}{16}, \frac{7\pi}{16}, \frac{15\pi}{16}, \frac{23\pi}{16}\}$

$r = (\sqrt{16^2+(-16)^2})^{1/4}$

$r = (\sqrt{512})^{1/4}$

$r = (16\sqrt{2})^{1/4}$

$r = 2(2)^{1/4}$

$z = \{2(2)^{1/4}e^{i(-\frac{\pi}{16})}, 2(2)^{1/4}e^{i(\frac{7\pi}{16})},  2(2)^{1/4}e^{i(\frac{15\pi}{16})},  2(2)^{1/4}e^{i(\frac{23\pi}{16})} \}$
\end{question}
\begin{question}{2f}
$z^6 = -1 - \sqrt{3}i$

$\theta = tan^{-1}(\sqrt(3)) = \frac{\pi}{3}$

$6\theta = [0, 12\pi]$

$6\theta = \{\frac{\pi}{3}, 2\pi+ \frac{\pi}{3}, 4\pi+\frac{\pi}{3}, 6\pi+\frac{\pi}{3}, 8\pi+\frac{\pi}{3}, 10\pi+\frac{\pi}{3}  \}$

$6\theta = \{\frac{\pi}{3}, \frac{7\pi}{3}, \frac{13\pi}{3}, \frac{19\pi}{3}, \frac{25\pi}{3}, \frac{31\pi}{3}  \}$

$\theta = \{\frac{\pi}{18}, \frac{7\pi}{18}, \frac{13\pi}{18}, \frac{19\pi}{18}, \frac{25\pi}{18}, \frac{31\pi}{18}  \}$

$r = (\sqrt{(-1)^2 + (-\sqrt{3})^2})^{1/6}$

$r = (\sqrt{1+3})^{1/6}$

$r = (2)^{1/6}$

$z = \{(2)^{1/6}e^{i\frac{\pi}{18}}, (2)^{1/6}e^{i\frac{7\pi}{18}}, (2)^{1/6}e^{i\frac{13\pi}{18}}, (2)^{1/6}e^{i\frac{19\pi}{18}}, (2)^{1/6}e^{i\frac{25\pi}{18}}, (2)^{1/6}e^{i\frac{31\pi}{18}}  \}$

\end{question}
\begin{question}{2g}
$z^2 + (4-4\sqrt{3}i)z -16 = 0$

Let $b = (4-4\sqrt{3}i)$

$z^2 + bz -16 = 0$

\begin{align*}
\frac{-b\pm \sqrt{b^2-4(-16)}}{2} 
  &= \frac{-b}{2}\pm \frac{\sqrt{b^2+64}}{2} \\
  &= \frac{-(4-4\sqrt{3}i)}{2}\pm \frac{\sqrt{(4-4\sqrt{3}i)^2+64}}{2} \\
  &= 2+2\sqrt{3}i  \pm  \frac{\sqrt{(-32-32\sqrt(3)i)+64}}{2} \\
  &= 2+2\sqrt{3}i  \pm  \frac{\sqrt{32-32\sqrt(3)i}}{2} \\
\end{align*}

\end{question}
% --------------------------------------------------------------
%                         Q3
% --------------------------------------------------------------
\begin{question}{3}
Suppose $(D, +, x)$ is an integral domain that is not a field.
\end{question}
\begin{question}{3a}
Prove that the ring $(D[x], +, x)$ of polynomials in the variable $x$ with coeffcients in $D$ is an integral domain. (For this question, you should assume that if $R$ is a commutative ring, then $R[x]$ is a commuative ring.)

\begin{proof}
$D[x]$ is a integral domain iff, $(D[x], +, x)$ is a commutative ring and $(D[x]^x, x)$ is closed.

Given that $D[x]$ is an assumed commuative ring.

When $f(x)g(x) \in D[x]$  $\setminus\{0\} \implies$ deg $f(x) = n\in \N$ and deg $g(x) = m \in \N$

By Lemma 6.1 in the notes, deg($f(x)g(x)$)$=n+m \in \N \implies f(x)g(x) \neq 0$.

Therefore, $D[x]$ is closed.

Because $D[x]$ is closed and an assumed commuative ring, $D[x]$ is a integral domain.

\end{proof}
\end{question}

\begin{question}{3b}
Let $F$ be a field. Explain why the polynomial ring $F[x,y]$ is an integral domain

\begin{proof}
A similar proof given in 3a can be used to prove $F[x,y]$.

Given that F is a field, it inherits the commutative ring property.

When $f(x,y)g(x,y) \in F[x,y]$  $\setminus\{0\} \implies$ deg $f(x,y) = n\in \N$ and deg $g(x,y) = m \in \N$

By Lemma 6.1 in the notes, deg($f(x,y)g(x,y)$)$=n+m \in \N \implies f(x,y)g(x,y) \neq 0$.

Therefore, $F[x,y]$ is closed.

Because $F[x,y]$ is closed and a commuative ring, $F[x,y]$ is a integral domain.
\end{proof}
\end{question}
\pagebreak
% --------------------------------------------------------------
%                         Q4
% --------------------------------------------------------------
\begin{question}{4}
For each of the following, find the unique quotient $q(x)$ and remainder r(x) in the given polynomial ring when $g(x)$ is divided by $f(x)$.
\end{question}
\begin{question}{4a}
$g(x) = x^4 + x^3 + x^2 -1$ and $f(x)=x^2+x+2$ in $\Z_3[x]$
\vspace{10cm}
\end{question}
\begin{question}{4b}
$g(x) = x^6 + 3x^5 + 4x^2 +2x + 2$ and $f(x)=x^2+2x+2$ in $\Z_5[x]$
\vspace{10cm}
\end{question}
\begin{question}{4c}
$g(x) = x^6 + 3x^5 + 4x^2 +2x +2$ and $f(x)=x^2+2x+2$ in $\Z_{11}[x]$
\vspace{10cm}
\end{question}
% --------------------------------------------------------------
%                         Q5
% --------------------------------------------------------------
\begin{question}{5}
Prove the $Remainder$ $Theorem$. Let $F$ be a field. Let $f(x) \in F[x]$ and let $c \in F$. Prove that the remainder of $f(x)$ divided by $(x-c)$ is $f(c)$.

\begin{proof}
Let $q(x)$ be the quotient, and $r(x)$ the remainder.

By division algorithm, $f(x) = q(x)(x-c) + r(x)$

Let $x = c$

$f(c) = q(c)(c-c) + r(c)$

$f(c) = q(c)0 + r(c)$

$f(c) = r(c)$

Hence $f(c)$ is the remainder, $r(c)$.

\end{proof}
\end{question}
\end{document}
