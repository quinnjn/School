%\documentclass{article}
%\usepackage[utf8]{inputenc}

%\title{a}
%\author{Quinn Neumiiller}
%\date{September 2013}

%\begin{document}

%\maketitle

%\section{Introduction}

%\end{document}

% --------------------------------------------------------------
% This is all preamble stuff that you don't have to worry about.
% Head down to where it says "Start here"
% --------------------------------------------------------------
 
\documentclass[12pt]{article}
 
\usepackage[margin=1in]{geometry} 
\usepackage{amsmath,amsthm,amssymb}
\usepackage{bbm}
 
\newcommand{\N}{\mathbb{N}}
\newcommand{\Z}{\mathbb{Z}}
\newcommand{\R}{\mathbb{R}}
\newcommand{\Q}{\mathbb{Q}}

 
\newenvironment{theorem}[2][Theorem]{\begin{trivlist}
\item[\hskip \labelsep {\bfseries #1}\hskip \labelsep {\bfseries #2.}]}{\end{trivlist}}
\newenvironment{lemma}[2][Lemma]{\begin{trivlist}
\item[\hskip \labelsep {\bfseries #1}\hskip \labelsep {\bfseries #2.}]}{\end{trivlist}}
\newenvironment{exercise}[2][Exercise]{\begin{trivlist}
\item[\hskip \labelsep {\bfseries #1}\hskip \labelsep {\bfseries #2.}]}{\end{trivlist}}
\newenvironment{problem}[2][Problem]{\begin{trivlist}
\item[\hskip \labelsep {\bfseries #1}\hskip \labelsep {\bfseries #2.}]}{\end{trivlist}}
\newenvironment{question}[2][Question]{\begin{trivlist}
\item[\hskip \labelsep {\bfseries #1}\hskip \labelsep {\bfseries #2.}]}{\end{trivlist}}
\newenvironment{corollary}[2][Corollary]{\begin{trivlist}
\item[\hskip \labelsep {\bfseries #1}\hskip \labelsep {\bfseries #2.}]}{\end{trivlist}}
 
\begin{document}

% --------------------------------------------------------------
%                         Samples
% --------------------------------------------------------------


%\begin{theorem}{x.yz} %You can use theorem, exercise, problem, or question here.  Modify x.yz to be whatever number you are proving
%Delete this text and write theorem statement here.
%\end{theorem}

%Blah, blah, blah. \mathbb{Z} Here is an example of the \texttt{align} environment:
%%Note 1: The * tells LaTeX not to number the lines.  If you remove the *, be sure to remove it below, too.
%%Note 2: Inside the align environment, you do not want to use $-signs.  The reason for this is that this is already a math environment. This is why we have to include \text{} around any text inside the align environment.
%\begin{align*}
%\sum_{i=1}^{k+1}i & = \left(\sum_{i=1}^{k}i\right) +(k+1)\\ 
%& = \frac{k(k+1)}{2}+k+1 & (\text{by inductive hypothesis})\\
%& = \frac{k(k+1)+2(k+1)}{2}\\
%& = \frac{(k+1)(k+2)}{2}\\
%& = \frac{(k+1)((k+1)+1)}{2}.
%\end{align*}
%\end{proof}
% 
%\begin{theorem}{x.yz}
%Let $n\in \Z$.  Then yada yada.
%\end{theorem}
% 
%\begin{proof}
%Blah, blah, blah.  I'm so smart.
%\end{proof}

% --------------------------------------------------------------
%                         Start here
% --------------------------------------------------------------
 
\title{Math 223 \\
Assignment 2 Algebraic Systems, Groups, Rings, and Fields
}
\author{Quinn Neumiiller} %if necessary, replace with your course title
 
\maketitle

% --------------------------------------------------------------
%                         Q1
% --------------------------------------------------------------

\begin{question}{1}
Let $A$ be a set with binary operations $\odot$. Supposed $\odot$ is associative and has identity.
\end{question}

\begin{question}{1(a)}
Invent some suitable but weird notation for the $\odot$-inverse on an element $x$ of $A$, other than $-x$ or $x^{-1}$. (Warning: It is pretty hard to get this wrong.)
\end{question}

$x^{\copyright}$

\begin{question}{1(b)}
Suppose $x$ and $y$ are elements of $A$ that have $\odot$-inverses. Simplify the inverse of $x\odot y$ using the notation you introduced in part(a)
\end{question}

${(x\odot y)}^{\copyright}$ = $y^{\copyright}\odot x^{\copyright}$


% --------------------------------------------------------------
%                         Q2
% --------------------------------------------------------------

\begin{question}{2}
Let A be a set with 4 elements
\end{question}

\begin{question}{2a}
How many binary operations of $A$ are possible?
\end{question}

TODO

\begin{question}{2b}
Give the operation table for a binary operation on $A=\{a,b,c,d\}$ that has an identity and has inverses, but is not commutative. Show that your operation is not associative.
\end{question}

\begin{tabular}{l|llll}
$\star$ & a & b & c & d \\
\hline
a       & a & b & c & d \\
b       & a & a & c & d \\
c       & a & b & a & d \\
d       & a & b & c & a
\end{tabular}


% --------------------------------------------------------------
%                         Q3
% --------------------------------------------------------------


\begin{question}{3}
Device whether or not each set is a group with respect to the indicated operation. If it is a group, just say "yes it is a group". If it is not a group, find the first of the four group properties (closed, associative, identity, inverses) that does not hold, and show that it does not hold.
\end{question}

\begin{question}{3a}
$\R$ - $\Q$ under addition.
\end{question}

\underline{Closed?} \textbf{False. $0 \notin (\R-\Q)$}

$-\sqrt{2} + \sqrt{2} = 0$. 


\begin{question}{3b}
$\Z^+$ under multiplication.
\end{question}

\underline{Inverse?} \textbf{False. $\frac{1}{y} \notin \Z^+$.}

$\forall z,y \in \Z^+$ : $z * y = e = y * z$

$z * y = 1 = y * z$

$y = \frac{1}{z}$


\begin{question}{3c}
the half-open interval (0,1] under multiplication.
\end{question}

\underline{Inverse?} \textbf{False. $\frac{1}{0.5} \notin (0,1]$, because $\frac{1}{0.5} > 1$}

$\forall z,y \in (0,1]$ : $z * y = e = y * z$

$z=0.5$

$z * y = 1 = y * z$

$y = \frac{1}{0.5}$


\begin{question}{3d}
the set of even integers with respect to addition
\end{question}

\textbf{yes it is a group}

\begin{question}{3e}
the set of even integers with respect to multiplication
\end{question}


\begin{question}{3f}
the set of positive divisors of 10 with respect to GCD.
\end{question}


\begin{question}{3g}
the set of surjective functions from $A$ to $A$ with respect to composition.
\end{question}


\begin{question}{3h}
the set of 2 x 2 matricies of the form $\begin{bmatrix}
  x & -y \\
  y & x  \\
 \end{bmatrix}$ satisifying $x^2 + y^2 = 1$ under matrix multiplication.

\end{question}


\begin{question}{3i}
the set of polynomials with real coefficients having constant term 1 under multiplication.
\end{question}


\begin{question}{3j}
the set of permutations $f$ of \{1,2,3\} with f(2) = 1 under composition.
\end{question}

% --------------------------------------------------------------
%                         Q4
% --------------------------------------------------------------

\begin{question}{4}
Decide if the following sets with two operations are rings. If it is not, find the first property of the ring defintiion that fails to hold. If it is a ring, also determine if it is an integral domain or a field.
\end{question}

\begin{question}{4a}
The set of nonnegative real numbers with respect to + and $\times$
\end{question}

\begin{question}{4b}
The set of even integers with respect to + and $\times$
\end{question}

\begin{question}{4c}
The set $D = \{x+y\sqrt{2} : x, y \in \Z\}$ under + and $\times$
\end{question}

\begin{question}{4d}
The set of functions $\R$ to $\R$ under + of values and function composition $\bigcirc$
\end{question}

\begin{question}{4e}
The set of 2 x 2 matricies of the form $\begin{bmatrix}
  x & -y \\
  y & x  \\
 \end{bmatrix}$ with $x,y \in \R$ under matrix addition and matrix multiplication. 
\end{question}

% --------------------------------------------------------------
%                         End Document
% --------------------------------------------------------------
\end{document}