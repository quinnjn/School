%\documentclass{article}
%\usepackage[utf8]{inputenc}

%\title{a}
%\author{Quinn Neumiiller}
%\date{September 2013}

%\begin{document}

%\maketitle

%\section{Introduction}

%\end{document}

% --------------------------------------------------------------
% This is all preamble stuff that you don't have to worry about.
% Head down to where it says "Start here"
% --------------------------------------------------------------
 
\documentclass[12pt]{article}
 
\usepackage[margin=1in]{geometry} 
\usepackage{amsmath,amsthm,amssymb}
%\usepackage{bbm}


\usepackage{fancyhdr}
\usepackage[us,12hr]{datetime} % `us' makes \today behave as usual in TeX/LaTeX
\fancypagestyle{plain}{
\fancyhf{}
\rfoot{Compiled on {\ddmmyyyydate\today} at \currenttime}
\lfoot{Page \thepage}
\renewcommand{\headrulewidth}{0pt}}
\pagestyle{plain}

 
\newcommand{\N}{\mathbb{N}}
\newcommand{\Z}{\mathbb{Z}}
\newcommand{\R}{\mathbb{R}}
\newcommand{\Q}{\mathbb{Q}}
\newcommand{\C}{\mathbb{C}}

 
\newenvironment{theorem}[2][Theorem]{\begin{trivlist}
\item[\hskip \labelsep {\bfseries #1}\hskip \labelsep {\bfseries #2.}]}{\end{trivlist}}
\newenvironment{lemma}[2][Lemma]{\begin{trivlist}
\item[\hskip \labelsep {\bfseries #1}\hskip \labelsep {\bfseries #2.}]}{\end{trivlist}}
\newenvironment{exercise}[2][Exercise]{\begin{trivlist}
\item[\hskip \labelsep {\bfseries #1}\hskip \labelsep {\bfseries #2.}]}{\end{trivlist}}
\newenvironment{problem}[2][Problem]{\begin{trivlist}
\item[\hskip \labelsep {\bfseries #1}\hskip \labelsep {\bfseries #2.}]}{\end{trivlist}}
\newenvironment{question}[2][Question]{\begin{trivlist}
\item[\hskip \labelsep {\bfseries #1}\hskip \labelsep {\bfseries #2.}]}{\end{trivlist}}
\newenvironment{corollary}[2][Corollary]{\begin{trivlist}
\item[\hskip \labelsep {\bfseries #1}\hskip \labelsep {\bfseries #2.}]}{\end{trivlist}}
 
\begin{document}

% --------------------------------------------------------------
%                         Samples
% --------------------------------------------------------------
 
\title{Math 223 \\
Assignment 7: Examples of Groups
}
\author{Quinn Neumiiller}
 
\maketitle

% --------------------------------------------------------------
%                         Q1
% --------------------------------------------------------------

\begin{question}{Q1}
Perform the following calculations in $\C$:
\end{question}

\begin{question}{1a}
\begin{align*}
(2-3i)+(-1 +4i) &= (2-3i)+(-1 +4i)\\
&= 2 -3i -1 + 4i\\
&= 1 + i
\end{align*}
\end{question}

\begin{question}{1b}
\begin{align*}
(2-11i)(-4+3i) 
&= -8 + 6i + 44i - 33i^2\\ 
&= 50i + 33 -8 \\
&= 50i +25 
\end{align*}
\end{question}
\begin{question}{1c}
\begin{align*}
(4+3i)^{-1}
\end{align*}
\end{question}
\begin{question}{1d}
\begin{align*}
(\sqrt{2}i)^{8}
 &= (\sqrt{2})^8(i)^8\\ 
 &= ((\sqrt{2})^2)^4((i)^2)^4\\ 
 &= (2)^4(-1)^4\\ 
 &= 16
\end{align*}
\end{question}
\begin{question}{1e}
\begin{align*}
(2+3i)\overline{(3+4i)}
 &= (2+3i)(3-4i)\\ 
 &= 6-8i+9i-12i^2\\ 
 &= 6+i+12\\ 
 &= 18+i\\ 
\end{align*}
\end{question}
\begin{question}{1f}
\begin{align*}
(2\sqrt{3}i-2)^{6}
 &= \\ 
\end{align*}
\end{question}
\begin{question}{1g}
\begin{align*}
(1-i)^{4}
 &= \\ 
\end{align*}
\end{question}
\begin{question}{1h}
\begin{align*}
(\frac{1}{4\sqrt{2}} - \frac{\sqrt{3i}}{4\sqrt{2}})^{9}
 &= \\ 
\end{align*}
\end{question}
\begin{question}{1i}
\begin{align*}
(re^{i\theta})^{-1}
 &= \\ 
\end{align*} 
Express in polar form
\end{question}
\begin{question}{1j}
\begin{align*}
(\sqrt{5}e^{\frac{2\pi i}{5}})(2\sqrt{5}e^{\frac{3\pi i}{5}})
 &= \\ 
\end{align*}
Express in standard $x+yi$ form
\end{question}
% --------------------------------------------------------------
%                         Q2
% --------------------------------------------------------------
\begin{question}{Q2}
Solve the following equations. Find all solutions for $z$ in $\C$. Express your answer in $x+yi$ form when convenient, otherwise leave them in polar form.
\end{question}

\begin{question}{2a}
\end{question}
\begin{question}{2b}
\end{question}
\begin{question}{2c}
\end{question}
\begin{question}{2d}
\end{question}
\begin{question}{2e}
\end{question}
\begin{question}{2f}
\end{question}
\begin{question}{2g}
\end{question}
% --------------------------------------------------------------
%                         Q3
% --------------------------------------------------------------
\begin{question}{3}
Suppose $(D, +, x)$ is an integral domain that is not a field.
\end{question}
\begin{question}{3a}
Prove that the ring $(D[x], +, x)$ of polynomials in the variable $x$ with coeffcients in $D$ is an integral domain. (For this question, you should assume that if $R$ is a commutative ring, then $R[x]$ is a commuative ring.)
\end{question}

\begin{question}{3b}
\end{question}
% --------------------------------------------------------------
%                         Q4
% --------------------------------------------------------------
\begin{question}{4}
For each of the following, find the unique quotient $q(x)$ and remainder r(x) in the given polynomial ring when $g(x)$ is divided by $f(x)$.
\end{question}
\begin{question}{4a}
\end{question}
\begin{question}{4b}
\end{question}
\begin{question}{4c}
\end{question}
% --------------------------------------------------------------
%                         Q5
% --------------------------------------------------------------
\begin{question}{5}
Prove the $Remainder$ $Theorem$. Let $F$ be a field. Let $f(x) \in F[x]$ and let $c \in F$. Prove that the remainder of $f(x)$ divided by $(x-c)$ is $f(c)$.
\end{question}
\end{document}
\item[\hskip \labelsep {\bfseries #1}\hskip \labelsep {\bfseries #2.}]}{\end{trivlist}}
\newenvironment{corollary}[2][Corollary]{\begin{trivlist}
\item[\hskip \labelsep {\bfseries #1}\hskip \labelsep {\bfseries #2.}]}{\end{trivlist}}
 
\begin{document}

% --------------------------------------------------------------
%                         Samples
% --------------------------------------------------------------
 
\title{Math 223 \\
Assignment 7: Examples of Groups
}
\author{Quinn Neumiiller}
 
\maketitle

% --------------------------------------------------------------
%                         Q1
% --------------------------------------------------------------

\begin{question}{Q1}
Perform the following calculations in $\C$:
\end{question}

\begin{question}{1a}
$(2-3i)+(-1 +4i)$
\end{question}
\begin{question}{1b}
$(2-11i)(-4+3i)$
\end{question}
\begin{question}{1c}
$(4+3i)^{-1}$
\end{question}
\begin{question}{1d}
$(\sqrt{2i})^{8}$
\end{question}
\begin{question}{1e}
$(2+3i)\overline{(3+4i)}$
\end{question}
\begin{question}{1f}
$(2\sqrt{3i}-2)^{6}$
\end{question}
\begin{question}{1g}
$(1-i)^{4}$
\end{question}
\begin{question}{1h}
$(\frac{1}{4\sqrt{2}} - \frac{\sqrt{3i}}{4\sqrt{2}})^{9}$
\end{question}
\begin{question}{1i}
$(re^{i\theta})^{-1}$ Express in polar form
\end{question}
\begin{question}{1j}
$(\sqrt{5}e^{\frac{2\pi i}{5}})(2\sqrt{5}e^{\frac{3\pi i}{5}})$
Express in standard $x+yi$ form
\end{question}
% --------------------------------------------------------------
%                         Q2
% --------------------------------------------------------------
\begin{question}{Q2}
Solve the following equations. Find all solutions for $z$ in $\C$. Express your answer in $x+yi$ form when convenient, otherwise leave them in polar form.
\end{question}

\begin{question}{2a}
\end{question}
\begin{question}{2b}
\end{question}
\begin{question}{2c}
\end{question}
\begin{question}{2d}
\end{question}
\begin{question}{2e}
\end{question}
\begin{question}{2f}
\end{question}
\begin{question}{2g}
\end{question}
% --------------------------------------------------------------
%                         Q3
% --------------------------------------------------------------
\begin{question}{3}
Suppose $(D, +, x)$ is an integral domain that is not a field.
\end{question}
\begin{question}{3a}
Prove that the ring $(D[x], +, x)$ of polynomials in the variable $x$ with coeffcients in $D$ is an integral domain. (For this question, you should assume that if $R$ is a commutative ring, then $R[x]$ is a commuative ring.)
\end{question}

\begin{question}{3b}
\end{question}
% --------------------------------------------------------------
%                         Q4
% --------------------------------------------------------------
\begin{question}{4}
For each of the following, find the unique quotient $q(x)$ and remainder r(x) in the given polynomial ring when $g(x)$ is divided by $f(x)$.
\end{question}
\begin{question}{4a}
\end{question}
\begin{question}{4b}
\end{question}
\begin{question}{4c}
\end{question}
% --------------------------------------------------------------
%                         Q5
% --------------------------------------------------------------
\begin{question}{5}
Prove the $Remainder$ $Theorem$. Let $F$ be a field. Let $f(x) \in F[x]$ and let $c \in F$. Prove that the remainder of $f(x)$ divided by $(x-c)$ is $f(c)$.
\end{question}
\end{document}
