%\documentclass{article}
%\usepackage[utf8]{inputenc}

%\title{a}
%\author{Quinn Neumiiller}
%\date{September 2013}

%\begin{document}

%\maketitle

%\section{Introduction}

%\end{document}

% --------------------------------------------------------------
% This is all preamble stuff that you don't have to worry about.
% Head down to where it says "Start here"
% --------------------------------------------------------------
 
\documentclass[12pt]{article}
 
\usepackage[margin=1in]{geometry} 
\usepackage{amsmath,amsthm,amssymb}
\usepackage{bbm}


\usepackage{fancyhdr}
\usepackage[us,12hr]{datetime} % `us' makes \today behave as usual in TeX/LaTeX
\fancypagestyle{plain}{
\fancyhf{}
\rfoot{Compiled on {\ddmmyyyydate\today} at \currenttime}
\lfoot{Page \thepage}
\renewcommand{\headrulewidth}{0pt}}
\pagestyle{plain}

 
\newcommand{\N}{\mathbb{N}}
\newcommand{\Z}{\mathbb{Z}}
\newcommand{\R}{\mathbb{R}}
\newcommand{\Q}{\mathbb{Q}}

 
\newenvironment{theorem}[2][Theorem]{\begin{trivlist}
\item[\hskip \labelsep {\bfseries #1}\hskip \labelsep {\bfseries #2.}]}{\end{trivlist}}
\newenvironment{lemma}[2][Lemma]{\begin{trivlist}
\item[\hskip \labelsep {\bfseries #1}\hskip \labelsep {\bfseries #2.}]}{\end{trivlist}}
\newenvironment{exercise}[2][Exercise]{\begin{trivlist}
\item[\hskip \labelsep {\bfseries #1}\hskip \labelsep {\bfseries #2.}]}{\end{trivlist}}
\newenvironment{problem}[2][Problem]{\begin{trivlist}
\item[\hskip \labelsep {\bfseries #1}\hskip \labelsep {\bfseries #2.}]}{\end{trivlist}}
\newenvironment{question}[2][Question]{\begin{trivlist}
\item[\hskip \labelsep {\bfseries #1}\hskip \labelsep {\bfseries #2.}]}{\end{trivlist}}
\newenvironment{corollary}[2][Corollary]{\begin{trivlist}
\item[\hskip \labelsep {\bfseries #1}\hskip \labelsep {\bfseries #2.}]}{\end{trivlist}}
 
\begin{document}

% --------------------------------------------------------------
%                         Samples
% --------------------------------------------------------------


%\begin{theorem}{x.yz} %You can use theorem, exercise, problem, or question here.  Modify x.yz to be whatever number you are proving
%Delete this text and write theorem statement here.
%\end{theorem}

%Blah, blah, blah. \mathbb{Z} Here is an example of the \texttt{align} environment:
%%Note 1: The * tells LaTeX not to number the lines.  If you remove the *, be sure to remove it below, too.
%%Note 2: Inside the align environment, you do not want to use $-signs.  The reason for this is that this is already a math environment. This is why we have to include \text{} around any text inside the align environment.
%\begin{align*}
%\sum_{i=1}^{k+1}i & = \left(\sum_{i=1}^{k}i\right) +(k+1)\\ 
%& = \frac{k(k+1)}{2}+k+1 & (\text{by inductive hypothesis})\\
%& = \frac{k(k+1)+2(k+1)}{2}\\
%& = \frac{(k+1)(k+2)}{2}\\
%& = \frac{(k+1)((k+1)+1)}{2}.
%\end{align*}
%\end{proof}
% 
%\begin{theorem}{x.yz}
%Let $n\in \Z$.  Then yada yada.
%\end{theorem}
% 
%\begin{proof}
%Blah, blah, blah.  I'm so smart.
%\end{proof}

% --------------------------------------------------------------
%                         Start here
% --------------------------------------------------------------
 
\title{Math 223 \\
Assignment 2 \underline{Episode 2}: The Algebraic Systems, Groups, Rings, and Fields Strikes Back
}
\author{Quinn Neumiiller} %if necessary, replace with your course title
 
\maketitle

% --------------------------------------------------------------
%                         Q1
% --------------------------------------------------------------

\begin{question}{1}
Divisibility Rules. Suppose that $a \in \N$ is written as $a = a_na_{n-1}...a_1a_0$ in base 10 notation, where each $a_{i}$ is a digit among the numbers 0 to 9.
\end{question}


\begin{question}{1a}
Prove that $a\equiv$ ($\displaystyle\sum\limits_{i=0}^n a_i$) mod 9.
\end{question}

\begin{question}{1b}
Prove that $a\equiv$ ($\displaystyle\sum\limits_{i=0}^n a_i$) mod 3.
\end{question}

\begin{question}{1c}
Prove that $a\equiv$ $a_0$ mod 5.

\end{question}

\begin{question}{1d}
Prove that $a\equiv$ ($\displaystyle\sum\limits_{i=0}^n (-1)^i a_i$) mod 11.
\end{question}

\begin{question}{1e}
Prove that modulo 7, $a \equiv (a_0 + 3a_1 + 3a_2)-(a_3+3a_4+2a_5)+(a_6+3a_7+2a_8)-(a_9+3a_{10}+2a_{11})+$...
\end{question}


% --------------------------------------------------------------
%                         Q2
% --------------------------------------------------------------

\begin{question}{2}
Find all solutions, if any, for the following euqations in $\Z_n$
\end{question}


%Steps to find this

% ax = b in Z_z
% ax = b mod z

%1. Determine if there is a solution
%
% gcd(a,z) = d
%
%  1. d=1 			One Solution
%  2. d>1, d|b		d 	Solutions
%  3. d>1, !(d|b)	No Solution

% z = a(z/a) + (z%a)
% continues until (z%a) = 1 or 0

%2. 

\begin{question}{2a}
$2x = 5$ mod $15$

1. Determine gcd(a,z)

gcd(2,15)

\begin{align*}
15 =& 2(7) + 1
=& 1
\end{align*}

2. Back substitute

\begin{align*}
1 = 15 - 2(7)\\
\end{align*}

3. Finding inverse
\begin{align*}
1 = 15 + 2(-7)\\
\end{align*} 

Inverse is -7

4. Solving for x
\begin{align*}
2x(-7) = 5(-7) in \Z_{15}\\
(-14)x = -35 in \Z_{15}\\
x = -35 in \Z_{15}\\
x = -35 in \Z_{15}\\
x = 10 in \Z_{15}\\
\end{align*}
\end{question}

\begin{question}{2b}
$23x = 1$ in $\Z_{41}$

1. Determine gcd(a,z)

gcd(23,41)

\begin{align*}
41 =& 23(1) +18 \\
23 =& 18(1) +5 \\
18 =& 5(3) + 3 \\
5 =& 3(1) +2 \\
3 =& 2(1) +1 \\
=& 1 \\
\end{align*}

There is one unique solution for x.

2. Back substitute

\begin{align*}
1 =& 3 - 2(1) \\
=& 3-(5-3) = 3(2)-5\\
=& (18-5(3))(2)-5 = 18(2)-5(7)\\ 
=& 18(2)-(23-18)(7) = 18(9)-23(7)\\
=& (41-23)(9)-23(7) = 41(9)-23(16)\\
=& 41(9)-23(16)\\ 
\end{align*}

3. Finding inverse
\begin{align*}
1 = 41(9) + 23x(-16)
\end{align*}
Inverse is -16

4. Solving for x
\begin{align*}
23x(-16) =& 1(-16) in \Z_{41}\\
1x =& -16 in \Z_{41}\\
x =& -16 in \Z_{41}
\end{align*}
\end{question}

\begin{question}{2c}
$1426x = 597$ in $\Z_{2000}$

1. Determine gcd(a,z)

gcd(1426,2000)

\begin{align*}
2000 =& 1426(1) + 574 \\
1426 =& 574(2) + 278 \\
574 =& 278(2) + 18 \\
278 =& 18(15) + 8 \\
18 =& 8(2) + 2 \\
8 =& 2(4) + 0
\end{align*}

gcd(1426,2000) = 2

Is $597|2$? No. There is no solution.
\end{question}

\begin{question}{2d}
$1731x = 871$ in $\Z_{2000}$

1. Determine gcd(a,z)

gcd(1731,2000)

\begin{align*}
2000 =& 1731(1) + 269 \\
1731 =& 269(6) + 117 \\
269 =& 117(2) + 35 \\
117 =& 35(3) + 12 \\
35 =& 12(2) + 11 \\
12 =& 11(1) + 1 \\
\end{align*}

gcd(1731,2000) = 1

2. Back substitute

\begin{align*}
1 =& 12 - 11\\
1 =& 12(3) - 35\\
1 =& 117(3) - 35(10)\\
1 =& 117(23) - 269(10)\\
1 =& 1731(23) - 269(148)\\
1 =& 1731(171) - 2000(148)\\
\end{align*}

3. Finding inverse
\begin{align*}
1 = 1731x(171) + 2000(148)
\end{align*}

Inverse is 171. 

4. Solving for x
\begin{align*}
1731x(171) =& 871(171) in \Z{2000}\\
1731x(171) =& 1(171) in \Z{2000}\\
296001x = 148941 in \Z{2000}\\
x = 941 in \Z{2000}\\
\end{align*}
\end{question}

\begin{question}{2e}
The system 
\begin{tabular}{l}
$8x + 3y=9$\\
$6x + 5y=2$
\end{tabular}
in $\Z_{12}$.

1. Determine gcd(a,z)

det = $10x \equiv 1$ mod $12$
gcd(10, 12) = 2

There is no solution because $2\nmid 1$

\end{question}

\begin{question}{2f}
$x^4 + 3x^2 +10 = 0$ in $\Z_{11}$
\end{question}

\begin{tabular}{c|c}
 &$x^4 + 3x^2 +10 = 0$ in $\Z_{11}$\\
\hline
0 & 16\\
1 & 16\\
2 & 20\\
3 & 28\\
4 & 24\\
5 & 24\\
6 & 28\\
7 & 36\\
8 & 48\\
9 & 48
\end{tabular}

No solution.

\begin{question}{2g}
$x^2 \equiv 17$ in $\Z_{24}$

\begin{tabular}{c|c}
& $x^2 \equiv 17$ in $\Z_{24}$\\
\hline
0 & $0$\\
1 & $1$\\
2 & $4$\\
3 & $9$\\
4 & $16$\\
5 & $25 \equiv 1$\\
6 & $36 \equiv 12$\\
7 & $49 \equiv 1$\\
8 & $64 \equiv 16$\\
9 & $81 \equiv 9$\\
10 & $100 \equiv 4$\\
11 & $121 \equiv 1$\\
12 & $144 \equiv 0$\\
13 & $169 \equiv 1$\\
14 & $196 \equiv 4$\\
15 & $225 \equiv 9$\\
16 & $256 \equiv 16$\\
17 & $289 \equiv 1$\\
18 & $324 \equiv 12$\\
19 & $361 \equiv 1$\\
20 & $400 \equiv 16$\\
21 & $441 \equiv 9$\\
22 & $484 \equiv 4$\\
23 & $529 \equiv 1$\\


\end{tabular}
\end{question}


% --------------------------------------------------------------
%                         Q3
% --------------------------------------------------------------

\begin{question}{3}
Find all solutions for $x$, up to congruence. If there is more than one equation, then find all simultaneous solutions up to congruence.
\end{question}


\begin{question}{3a}
$x \equiv 1$ mod 4 and $x \equiv 7$ mod 13.
\end{question}

\begin{question}{3b}
$x \equiv 11$ mod 142 and $x \equiv 25$ mod 86.
\end{question}

\begin{question}{3c}
$x \equiv 2^{63}$ mod 61.
\end{question}

\begin{question}{3d}
$x \equiv 7^{78}$ mod 79.
\end{question}

\begin{question}{3e}
$x^2 + 3x \equiv 3$ mod 8.
\end{question}

% --------------------------------------------------------------
%                         Q4
% --------------------------------------------------------------

\begin{question}{4}
Suppose $p$ is a position prime integer. Prove that $\forall x,y \in \Z, (x+y)^p \equiv x^p + y^p$ mod $p$
\end{question}

% --------------------------------------------------------------
%                         Q5
% --------------------------------------------------------------

\begin{question}{5}
Let $\varphi$ : $\Z+ \rightarrow \Z^+$ be the Euler-phi function
\end{question}


\begin{question}{5a}
\begin{align*}
\phi(p^4q^2) =& \phi(p^4)\phi(q^2)\\
=& (p^{4-1}(p-1))(q^{2-1}(q-1))\\
=& (p^{3}(p-1))(q(q-1))
\end{align*}
\end{question}

\begin{question}{5b}
\begin{align*}
\phi(343000) =& \phi(343)\phi(1000)\\
=& \phi(7^3)\phi(2^3)\phi(5^3)\\
=& 7^2(7-1)) (2^2(2-1)) (5^2(5-1)\\
=& 7^2(6))*(2^2)*(5^2(4)\\
=& 49(6))*4*(25(4)\\
=& 294*4*100\\
=& 117,600
\end{align*}
\end{question}

\begin{question}{5c}
\begin{align*}
\phi(n) =& n^{1-1}(n-1)\\
=& n^{0}(n-1)\\
=& (n-1)
\end{align*}

$\phi(101)=100$
\end{question}

\begin{question}{5d}
\begin{align*}
\phi(2m) =& \phi(2) \phi(m)\\
=& 2^{1-1}(2-1) \phi(m)\\
=& (2-1) \phi(m)\\
=& 1\phi(m)\\
=& \phi(m)\\
\end{align*}
\end{question}





% --------------------------------------------------------------
%                         Q6
% --------------------------------------------------------------

\begin{question}{6a}
Give the multiplication table for the group $U_{12}$.
\end{question}

\begin{question}{6b}
Compute the inverse of 43 in $U_{63}$
\end{question}



\end{document}
