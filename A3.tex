%\documentclass{article}
%\usepackage[utf8]{inputenc}

%\title{a}
%\author{Quinn Neumiiller}
%\date{September 2013}

%\begin{document}

%\maketitle

%\section{Introduction}

%\end{document}

% --------------------------------------------------------------
% This is all preamble stuff that you don't have to worry about.
% Head down to where it says "Start here"
% --------------------------------------------------------------
 
\documentclass[12pt]{article}
 
\usepackage[margin=1in]{geometry} 
\usepackage{amsmath,amsthm,amssymb}
\usepackage{bbm}


\usepackage{fancyhdr}
\usepackage[us,12hr]{datetime} % `us' makes \today behave as usual in TeX/LaTeX
\fancypagestyle{plain}{
\fancyhf{}
\rfoot{Compiled on {\ddmmyyyydate\today} at \currenttime}
\lfoot{Page \thepage}
\renewcommand{\headrulewidth}{0pt}}
\pagestyle{plain}

 
\newcommand{\N}{\mathbb{N}}
\newcommand{\Z}{\mathbb{Z}}
\newcommand{\R}{\mathbb{R}}
\newcommand{\Q}{\mathbb{Q}}

 
\newenvironment{theorem}[2][Theorem]{\begin{trivlist}
\item[\hskip \labelsep {\bfseries #1}\hskip \labelsep {\bfseries #2.}]}{\end{trivlist}}
\newenvironment{lemma}[2][Lemma]{\begin{trivlist}
\item[\hskip \labelsep {\bfseries #1}\hskip \labelsep {\bfseries #2.}]}{\end{trivlist}}
\newenvironment{exercise}[2][Exercise]{\begin{trivlist}
\item[\hskip \labelsep {\bfseries #1}\hskip \labelsep {\bfseries #2.}]}{\end{trivlist}}
\newenvironment{problem}[2][Problem]{\begin{trivlist}
\item[\hskip \labelsep {\bfseries #1}\hskip \labelsep {\bfseries #2.}]}{\end{trivlist}}
\newenvironment{question}[2][Question]{\begin{trivlist}
\item[\hskip \labelsep {\bfseries #1}\hskip \labelsep {\bfseries #2.}]}{\end{trivlist}}
\newenvironment{corollary}[2][Corollary]{\begin{trivlist}
\item[\hskip \labelsep {\bfseries #1}\hskip \labelsep {\bfseries #2.}]}{\end{trivlist}}
 
\begin{document}

% --------------------------------------------------------------
%                         Samples
% --------------------------------------------------------------


%\begin{theorem}{x.yz} %You can use theorem, exercise, problem, or question here.  Modify x.yz to be whatever number you are proving
%Delete this text and write theorem statement here.
%\end{theorem}

%Blah, blah, blah. \mathbb{Z} Here is an example of the \texttt{align} environment:
%%Note 1: The * tells LaTeX not to number the lines.  If you remove the *, be sure to remove it below, too.
%%Note 2: Inside the align environment, you do not want to use $-signs.  The reason for this is that this is already a math environment. This is why we have to include \text{} around any text inside the align environment.
%\begin{align*}
%\sum_{i=1}^{k+1}i & = \left(\sum_{i=1}^{k}i\right) +(k+1)\\ 
%& = \frac{k(k+1)}{2}+k+1 & (\text{by inductive hypothesis})\\
%& = \frac{k(k+1)+2(k+1)}{2}\\
%& = \frac{(k+1)(k+2)}{2}\\
%& = \frac{(k+1)((k+1)+1)}{2}.
%\end{align*}
%\end{proof}
% 
%\begin{theorem}{x.yz}
%Let $n\in \Z$.  Then yada yada.
%\end{theorem}
% 
%\begin{proof}
%Blah, blah, blah.  I'm so smart.
%\end{proof}

% --------------------------------------------------------------
%                         Start here
% --------------------------------------------------------------
 
\title{Math 223 \\
Assignment 2 \underline{Episode 2}: The Algebraic Systems, Groups, Rings, and Fields Strikes Back
}
\author{Quinn Neumiiller} %if necessary, replace with your course title
 
\maketitle

% --------------------------------------------------------------
%                         Q1
% --------------------------------------------------------------

\begin{question}{1}
Divisibility Rules. Suppose that $a \in \N$ is written as $a = a_na_{n-1}...a_1a_0$ in base 10 notation, where each $a_{i}$ is a digit among the numbers 0 to 9.
\end{question}


\begin{question}{1a}
\end{question}

\begin{question}{1b}
\end{question}

\begin{question}{1c}
\end{question}

\begin{question}{1d}
\end{question}

\begin{question}{1e}
\end{question}


% --------------------------------------------------------------
%                         Q2
% --------------------------------------------------------------

\begin{question}{2}
Find all solutions, if any, for the following euqations in $\Z_n$
\end{question}


\begin{question}{2a}

\end{question}

\begin{question}{2b}

\end{question}

\begin{question}{2c}

\end{question}

\begin{question}{2d}

\end{question}

\begin{question}{2e}

\end{question}

\begin{question}{2f}

\end{question}

\begin{question}{2g}

\end{question}


% --------------------------------------------------------------
%                         Q3
% --------------------------------------------------------------

\begin{question}{3}
Find all solutions for $x$, up to congruence. If there is more than one equation, then find all simultaneous solutions up to congruence.
\end{question}


\begin{question}{3a}

\end{question}

\begin{question}{3b}

\end{question}

\begin{question}{3c}

\end{question}

\begin{question}{3d}

\end{question}

\begin{question}{3e}

\end{question}

% --------------------------------------------------------------
%                         Q4
% --------------------------------------------------------------

\begin{question}{4}
Suppose $p$ is a position prime integer. Prove that $\forall x,y \in \Z, (x+y)^p \equiv x^p + y^p$ mod $p$
\end{question}

% --------------------------------------------------------------
%                         Q5
% --------------------------------------------------------------

\begin{question}{5}
Let $\epsilon$ : $\Z+ \rightarrow \Z^+$ be the Euler-phi function
\end{question}


\begin{question}{5a}
\begin{align*}
$\phi(p^4q^2)$ =& $\phi(p^4)\phi(q^2)$\\
=&$ (p^{4-1}(p-1))(q^{2-1}(q-1))$\\
=&$ (p^{3}(p-1))(q(q-1))$\\
\end{align*}
\end{question}

\begin{question}{5b}
\begin{align*}

$\phi(343000)$ 
=& $\phi(343)\phi(1000)$\\
=& $\phi(7^3)\phi(2^3)\phi(5^3)$\\
=& $7^2(7-1)) (2^2(2-1)) (5^2(5-1)$\\
=& $7^2(6))*(2^2)*(5^2(4)$\\
=& $49(6))*4*(25(4)$\\
=& $294*4*100$\\
=& $117,600$

\end{align*}
\end{question}

\begin{question}{5c}
\begin{align*}

$\phi(n)$ 
=& $n^{1-1}(n-1)$\\
=& $n^{0}(n-1)$\\
=& $(n-1)$
\end{align*}

$\phi(101)=100$
\end{question}

\begin{question}{5d}

$\phi(2m)$
=& $\phi(2) \phi(m)$\\
=& $2^{1-1}(2-1) \phi(m)$\\
=& $(2-1) \phi(m)$\\
=& $1\phi(m)$\\
=& $\phi(m)$\\

\end{question}





% --------------------------------------------------------------
%                         Q6
% --------------------------------------------------------------

\begin{question}{6a}
Give the multiplication table for the group $U_{12}$.
\end{question}

\begin{question}{6b}
Compute the inverse of 43 in $U_{63}$
\end{question}



\end{document}
