%\documentclass{article}
%\usepackage[utf8]{inputenc}

%\title{a}
%\author{Quinn Neumiiller}
%\date{September 2013}

%\begin{document}

%\maketitle

%\section{Introduction}

%\end{document}

% --------------------------------------------------------------
% This is all preamble stuff that you don't have to worry about.
% Head down to where it says "Start here"
% --------------------------------------------------------------
 
\documentclass[12pt]{article}
 
\usepackage[margin=1in]{geometry} 
\usepackage{amsmath,amsthm,amssymb}
\usepackage{bbm}


\usepackage{fancyhdr}
\usepackage[us,12hr]{datetime} % `us' makes \today behave as usual in TeX/LaTeX
\fancypagestyle{plain}{
\fancyhf{}
\rfoot{Compiled on {\ddmmyyyydate\today} at \currenttime}
\lfoot{Page \thepage}
\renewcommand{\headrulewidth}{0pt}}
\pagestyle{plain}

 
\newcommand{\N}{\mathbb{N}}
\newcommand{\Z}{\mathbb{Z}}
\newcommand{\R}{\mathbb{R}}
\newcommand{\Q}{\mathbb{Q}}

 
\newenvironment{theorem}[2][Theorem]{\begin{trivlist}
\item[\hskip \labelsep {\bfseries #1}\hskip \labelsep {\bfseries #2.}]}{\end{trivlist}}
\newenvironment{lemma}[2][Lemma]{\begin{trivlist}
\item[\hskip \labelsep {\bfseries #1}\hskip \labelsep {\bfseries #2.}]}{\end{trivlist}}
\newenvironment{exercise}[2][Exercise]{\begin{trivlist}
\item[\hskip \labelsep {\bfseries #1}\hskip \labelsep {\bfseries #2.}]}{\end{trivlist}}
\newenvironment{problem}[2][Problem]{\begin{trivlist}
\item[\hskip \labelsep {\bfseries #1}\hskip \labelsep {\bfseries #2.}]}{\end{trivlist}}
\newenvironment{question}[2][Question]{\begin{trivlist}
\item[\hskip \labelsep {\bfseries #1}\hskip \labelsep {\bfseries #2.}]}{\end{trivlist}}
\newenvironment{corollary}[2][Corollary]{\begin{trivlist}
\item[\hskip \labelsep {\bfseries #1}\hskip \labelsep {\bfseries #2.}]}{\end{trivlist}}
 
\begin{document}

% --------------------------------------------------------------
%                         Samples
% --------------------------------------------------------------


%\begin{theorem}{x.yz} %You can use theorem, exercise, problem, or question here.  Modify x.yz to be whatever number you are proving
%Delete this text and write theorem statement here.
%\end{theorem}

%Blah, blah, blah. \mathbb{Z} Here is an example of the \texttt{align} environment:
%%Note 1: The * tells LaTeX not to number the lines.  If you remove the *, be sure to remove it below, too.
%%Note 2: Inside the align environment, you do not want to use $-signs.  The reason for this is that this is already a math environment. This is why we have to include \text{} around any text inside the align environment.
%\begin{align*}
%\sum_{i=1}^{k+1}i & = \left(\sum_{i=1}^{k}i\right) +(k+1)\\ 
%& = \frac{k(k+1)}{2}+k+1 & (\text{by inductive hypothesis})\\
%& = \frac{k(k+1)+2(k+1)}{2}\\
%& = \frac{(k+1)(k+2)}{2}\\
%& = \frac{(k+1)((k+1)+1)}{2}.
%\end{align*}
%\end{proof}
% 
%\begin{theorem}{x.yz}
%Let $n\in \Z$.  Then yada yada.
%\end{theorem}
% 
%\begin{proof}
%Blah, blah, blah.  I'm so smart.
%\end{proof}

% --------------------------------------------------------------
%                         Start here
% --------------------------------------------------------------
 
\title{Math 223 \\
Assignment 2 \underline{Episode 2}: The Algebraic Systems, Groups, Rings, and Fields Strikes Back
}
\author{Quinn Neumiiller} %if necessary, replace with your course title
 
\maketitle

% --------------------------------------------------------------
%                         Q1
% --------------------------------------------------------------

\begin{question}{1}
Divisibility Rules. Suppose that $a \in \N$ is written as $a = a_na_{n-1}...a_1a_0$ in base 10 notation, where each $a_{i}$ is a digit among the numbers 0 to 9.
\end{question}


\begin{question}{1a}
Prove that $a\equiv$ ($\displaystyle\sum\limits_{i=0}^n a_i$) mod 9.
\end{question}

\begin{question}{1b}
Prove that $a\equiv$ ($\displaystyle\sum\limits_{i=0}^n a_i$) mod 3.
\end{question}

\begin{question}{1c}
Prove that $a\equiv$ $a_0$ mod 5.

\end{question}

\begin{question}{1d}
Prove that $a\equiv$ ($\displaystyle\sum\limits_{i=0}^n (-1)^i a_i$) mod 11.
\end{question}

\begin{question}{1e}
Prove that modulo 7, $a \equiv (a_0 + 3a_1 + 3a_2)-(a_3+3a_4+2a_5)+(a_6+3a_7+2a_8)-(a_9+3a_{10}+2a_{11})+$...
\end{question}


% --------------------------------------------------------------
%                         Q2
% --------------------------------------------------------------

\begin{question}{2}
Find all solutions, if any, for the following euqations in $\Z_n$
\end{question}


%Steps to find this

% ax = b in Z_z
% ax = b mod z

%1. Determine if there is a solution
%
% gcd(a,b) = d
%
%  1. d=1 			One Solution
%  2. d>1, d|b		d 	Solutions
%  3. d>1, !(d|b)	No Solution

%2. 

\begin{question}{2a}
$2x = 5$ in $\Z_{15}$

$2x \equiv 5$ mod $15$\\



\end{question}

\begin{question}{2b}
$23x = 1 in \Z_{41}$
\end{question}

\begin{question}{2c}
$1426x = 597 in \Z_{2000}$
\end{question}

\begin{question}{2d}
$1731x = 871 in \Z_{2000}$
\end{question}

\begin{question}{2e}
The system 
\begin{tabular}{l}
$8x + 3y=9$\\
$6x + 5y=2$
\end{tabular}
in $\Z_{12}$.

\end{question}

\begin{question}{2f}
$x^4 + 3x^2 +10 = 0$ in $\Z_{11}$
\end{question}

\begin{question}{2g}
$x^2 = 17$ in $\Z_{24}$
\end{question}


% --------------------------------------------------------------
%                         Q3
% --------------------------------------------------------------

\begin{question}{3}
Find all solutions for $x$, up to congruence. If there is more than one equation, then find all simultaneous solutions up to congruence.
\end{question}


\begin{question}{3a}
$x \equiv 1$ mod 4 and $x \equiv 7$ mod 13.
\end{question}

\begin{question}{3b}
$x \equiv 11$ mod 142 and $x \equiv 25$ mod 86.
\end{question}

\begin{question}{3c}
$x \equiv 2^{63}$ mod 61.
\end{question}

\begin{question}{3d}
$x \equiv 7^{78}$ mod 79.
\end{question}

\begin{question}{3e}
$x^2 + 3x \equiv 3$ mod 8.
\end{question}

% --------------------------------------------------------------
%                         Q4
% --------------------------------------------------------------

\begin{question}{4}
Suppose $p$ is a position prime integer. Prove that $\forall x,y \in \Z, (x+y)^p \equiv x^p + y^p$ mod $p$
\end{question}

% --------------------------------------------------------------
%                         Q5
% --------------------------------------------------------------

\begin{question}{5}
Let $\varphi$ : $\Z+ \rightarrow \Z^+$ be the Euler-phi function
\end{question}


\begin{question}{5a}
If $p$ and $q$ are distinct primes, determine $\varphi(p^4q^2)$
\end{question}

\begin{question}{5b}
Compute $\varphi(343000)$
\end{question}

\begin{question}{5c}
Find an example of an integer $n$ for which $\varphi(n) = 100$.
\end{question}

\begin{question}{5d}
If $m$ is odd, show that $\varphi(2m) = \varphi(m).$
\end{question}





% --------------------------------------------------------------
%                         Q6
% --------------------------------------------------------------

\begin{question}{6a}
Give the multiplication table for the group $U_{12}$.
\end{question}

\begin{question}{6b}
Compute the inverse of 43 in $U_{63}$
\end{question}



\end{document}
