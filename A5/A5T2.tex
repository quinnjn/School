%\documentclass{article}
%\usepackage[utf8]{inputenc}

%\title{a}
%\author{Quinn Neumiiller}
%\date{September 2013}

%\begin{document}

%\maketitle

%\section{Introduction}

%\end{document}

% --------------------------------------------------------------
% This is all preamble stuff that you don't have to worry about.
% Head down to where it says "Start here"
% --------------------------------------------------------------
 
\documentclass[12pt]{article}
 
\usepackage[margin=1in]{geometry} 
\usepackage{amsmath,amsthm,amssymb}
\usepackage{bbm}


\usepackage{fancyhdr}
\usepackage[us,12hr]{datetime} % `us' makes \today behave as usual in TeX/LaTeX
\fancypagestyle{plain}{
\fancyhf{}
\rfoot{Compiled on {\ddmmyyyydate\today} at \currenttime}
\lfoot{Page \thepage}
\renewcommand{\headrulewidth}{0pt}}
\pagestyle{plain}

 
\newcommand{\N}{\mathbb{N}}
\newcommand{\Z}{\mathbb{Z}}
\newcommand{\R}{\mathbb{R}}
\newcommand{\Q}{\mathbb{Q}}

 
\newenvironment{theorem}[2][Theorem]{\begin{trivlist}
\item[\hskip \labelsep {\bfseries #1}\hskip \labelsep {\bfseries #2.}]}{\end{trivlist}}
\newenvironment{lemma}[2][Lemma]{\begin{trivlist}
\item[\hskip \labelsep {\bfseries #1}\hskip \labelsep {\bfseries #2.}]}{\end{trivlist}}
\newenvironment{exercise}[2][Exercise]{\begin{trivlist}
\item[\hskip \labelsep {\bfseries #1}\hskip \labelsep {\bfseries #2.}]}{\end{trivlist}}
\newenvironment{problem}[2][Problem]{\begin{trivlist}
\item[\hskip \labelsep {\bfseries #1}\hskip \labelsep {\bfseries #2.}]}{\end{trivlist}}
\newenvironment{question}[2][Question]{\begin{trivlist}
\item[\hskip \labelsep {\bfseries #1}\hskip \labelsep {\bfseries #2.}]}{\end{trivlist}}
\newenvironment{corollary}[2][Corollary]{\begin{trivlist}
\item[\hskip \labelsep {\bfseries #1}\hskip \labelsep {\bfseries #2.}]}{\end{trivlist}}
 
\begin{document}

% --------------------------------------------------------------
%                         Samples
% --------------------------------------------------------------


%\begin{theorem}{x.yz} %You can use theorem, exercise, problem, or question here.  Modify x.yz to be whatever number you are proving
%Delete this text and write theorem statement here.
%\end{theorem}

%Blah, blah, blah. \mathbb{Z} Here is an example of the \texttt{align} environment:
%%Note 1: The * tells LaTeX not to number the lines.  If you remove the *, be sure to remove it below, too.
%%Note 2: Inside the align environment, you do not want to use $-signs.  The reason for this is that this is already a math environment. This is why we have to include \text{} around any text inside the align environment.
%\begin{align*}
%\sum_{i=1}^{k+1}i & = \left(\sum_{i=1}^{k}i\right) +(k+1)\\ 
%& = \frac{k(k+1)}{2}+k+1 & (\text{by inductive hypothesis})\\
%& = \frac{k(k+1)+2(k+1)}{2}\\
%& = \frac{(k+1)(k+2)}{2}\\
%& = \frac{(k+1)((k+1)+1)}{2}.
%\end{align*}
%\end{proof}
% 
%\begin{theorem}{x.yz}
%Let $n\in \Z$.  Then yada yada.
%\end{theorem}
% 
%\begin{proof}
%Blah, blah, blah.  I'm so smart.
%\end{proof}

% --------------------------------------------------------------
%                         Start here
% --------------------------------------------------------------
 
\title{Math 223 \\
Assignment 5: Examples of Groups
}
\author{Quinn Neumiiller}
 
\maketitle

% --------------------------------------------------------------
%                         Q1
% --------------------------------------------------------------

\begin{question}{1b} Show that $a$ and $b^{-1}ab$ have the same order.

Claim: $|a| = |b^{-1}ab|$
\begin{proof}
Let $|a|=m$, $|b^{-1}ab|=n$

$a^m = e, (b^{-1}ab)^n = e$

$(b^{-1}ab)^n = e$

$b^{-1}a^nb = e$

$a^n = e$ Since the order of $a^m$ must be the least such exponent, $m \leq n$.


Alternatively, 

$a^m = e$

$b^{-1}(a^m) = b^{-1}(e)$

$(b^{-1}a^m) = b^{-1}$

$b(b^{-1}a^m) = b(b^{-1})$

$ba^mb^{-1} = e$ Since the order of $b^{-1}a^nb$ must be the least such exponent, $n \leq m$.

Since, $m \leq n$ and $n \leq m$ then $m = n$.
\end{proof}
\end{question}

\begin{question}{1c} Show that $ab$ and $ba$ have the same order.

Claim: $|ab| = |ba|$
\begin{proof}
Let $|ab|=m$, $|ba|=n$

$(ab)^m = e$

$a^-m((ab)^m) = a^-m(e)$

$b^m = a^-m$

$b^{-m}(b^m) = b^{-m}(a^-m)$

$e = (ab)^{-m}$

$(ba)^m(e) = (ba)^m((ab)^{-m})$

$(ba)^m = e$ Since the order of $ba$ must be the least such exponent, $n \leq m$.

Alternatively,

$(ba)^n = e$

$b^{-n}(b^na^n) = b^{-n}(e)$

$a^n = b^{-n}$

$a^{-n}(a^n) = a^{-n}(b^{-n})$

$e = b^{-n}a^{-n}$

$e = (ab)^{-n}$

$(ab)^n = e$ Since the order of $ab$ must be the least such exponent, $m \leq n$.

Since, $m \leq n$ and $n \leq m$ then $m = n$.
\end{proof}
\end{question}

% --------------------------------------------------------------
%                         Q2
% --------------------------------------------------------------

\begin{question}{2}
Suppose $a$ and $b$ are elements of the group ($\Z_n$, +) that have the same order. Show
that $\langle a \rangle$ = $\langle b \rangle$.

Suppose $c \in \langle a \rangle, \langle b \rangle$. Since $|a|$ = $|b|$, This implies that $\langle a \rangle \subseteq \langle b \rangle$ and $\langle a \rangle \subseteq \langle b \rangle$, and therefore $\langle a \rangle$ = $\langle b \rangle$.

\end{question}

\end{document}