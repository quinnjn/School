%\documentclass{article}
%\usepackage[utf8]{inputenc}

%\title{a}
%\author{Quinn Neumiiller}
%\date{September 2013}

%\begin{document}

%\maketitle

%\section{Introduction}

%\end{document}

% --------------------------------------------------------------
% This is all preamble stuff that you don't have to worry about.
% Head down to where it says "Start here"
% --------------------------------------------------------------
 
\documentclass[12pt]{article}
 
\usepackage[margin=1in]{geometry} 
\usepackage{amsmath,amsthm,amssymb}
\usepackage{bbm}


\usepackage{fancyhdr}
\usepackage[us,12hr]{datetime} % `us' makes \today behave as usual in TeX/LaTeX
\fancypagestyle{plain}{
\fancyhf{}
\rfoot{Compiled on {\ddmmyyyydate\today} at \currenttime}
\lfoot{Page \thepage}
\renewcommand{\headrulewidth}{0pt}}
\pagestyle{plain}

 
\newcommand{\N}{\mathbb{N}}
\newcommand{\Z}{\mathbb{Z}}
\newcommand{\R}{\mathbb{R}}
\newcommand{\Q}{\mathbb{Q}}

 
\newenvironment{theorem}[2][Theorem]{\begin{trivlist}
\item[\hskip \labelsep {\bfseries #1}\hskip \labelsep {\bfseries #2.}]}{\end{trivlist}}
\newenvironment{lemma}[2][Lemma]{\begin{trivlist}
\item[\hskip \labelsep {\bfseries #1}\hskip \labelsep {\bfseries #2.}]}{\end{trivlist}}
\newenvironment{exercise}[2][Exercise]{\begin{trivlist}
\item[\hskip \labelsep {\bfseries #1}\hskip \labelsep {\bfseries #2.}]}{\end{trivlist}}
\newenvironment{problem}[2][Problem]{\begin{trivlist}
\item[\hskip \labelsep {\bfseries #1}\hskip \labelsep {\bfseries #2.}]}{\end{trivlist}}
\newenvironment{question}[2][Question]{\begin{trivlist}
\item[\hskip \labelsep {\bfseries #1}\hskip \labelsep {\bfseries #2.}]}{\end{trivlist}}
\newenvironment{corollary}[2][Corollary]{\begin{trivlist}
\item[\hskip \labelsep {\bfseries #1}\hskip \labelsep {\bfseries #2.}]}{\end{trivlist}}
 
\begin{document}

% --------------------------------------------------------------
%                         Samples
% --------------------------------------------------------------


%\begin{theorem}{x.yz} %You can use theorem, exercise, problem, or question here.  Modify x.yz to be whatever number you are proving
%Delete this text and write theorem statement here.
%\end{theorem}

%Blah, blah, blah. \mathbb{Z} Here is an example of the \texttt{align} environment:
%%Note 1: The * tells LaTeX not to number the lines.  If you remove the *, be sure to remove it below, too.
%%Note 2: Inside the align environment, you do not want to use $-signs.  The reason for this is that this is already a math environment. This is why we have to include \text{} around any text inside the align environment.
%\begin{align*}
%\sum_{i=1}^{k+1}i & = \left(\sum_{i=1}^{k}i\right) +(k+1)\\ 
%& = \frac{k(k+1)}{2}+k+1 & (\text{by inductive hypothesis})\\
%& = \frac{k(k+1)+2(k+1)}{2}\\
%& = \frac{(k+1)(k+2)}{2}\\
%& = \frac{(k+1)((k+1)+1)}{2}.
%\end{align*}
%\end{proof}
% 
%\begin{theorem}{x.yz}
%Let $n\in \Z$.  Then yada yada.
%\end{theorem}
% 
%\begin{proof}
%Blah, blah, blah.  I'm so smart.
%\end{proof}

% --------------------------------------------------------------
%                         Start here
% --------------------------------------------------------------
 
\title{Math 223 \\
Assignment 5: Examples of Groups
}
\author{Quinn Neumiiller}
 
\maketitle

% --------------------------------------------------------------
%                         Q1
% --------------------------------------------------------------

\begin{question}{1}
Let $a$ and $b$ be elements of a multiplicative group $G$.
\begin{question}{1a}
Show that $a$ and $a^{-1}$ have the same order.

Claim:
$|a| = |a^{-1}|$

\begin{proof}
Let $|a| = m$, $|a^{-1}| = n$

By definition of the order of an element, $|a| = m$ means that $a^m = e$, and $m$ is the smallest positive integer with this result. 

$a^m = e$

$(a^{-m})a^m = (a^{-m})e$

$e = a^{-m}$

$e = (a^{-1})^m$. Since the order of $a^{-1}$ must be the least such exponent, $n \leq m$. 

Alternatively, 

$(a^{-1})^n = e$

$(a^n)(a^{-1})^n = e(a^n)$

$(a^n)(a^{-n}) = a^n$

$e = a^n$ Since the order of $a$ must be the least such exponent, $m \leq n$.

Since, $m \leq n$ and $n \leq m$ then $m = n$.
\end{proof}
\end{question}

\begin{question}{1b} Show that $a$ and $b^{-1}ab$ have the same order.

Claim: $|a| = |b^{-1}ab|$
\begin{proof}
Let $|a|=m$, $|b^{-1}ab|=n$

$a^m = e, (b^{-1}ab)^n = e$

$(b^{-1}ab)^n = e$

$b^n(b^{-n} a^n b^n ) = b^n(e)$

$ea^n b^n = b^n$

$a^n b^n = b^n$

$b^{-n}(a^n b^n) = b^{-n}(b^n)$

$a^n e = e$ Since the order of $a^m$ must be the least such exponent, $m \leq n$.


Alternatively, 

$a^m = e$

$b^{-n}(a^m) = b^{-n}(e)$

$(b^{-n}a^m) = b^{-n}$

$b^n(b^{-n}a^m) = b^n(b^{-n})$

$b^na^mb^{-n} = e$ Since the order of $b^{-n}ab^n$ must be the least such exponent, $n \leq m$.

Since, $m \leq n$ and $n \leq m$ then $m = n$.
\end{proof}
\end{question}

\begin{question}{1c} Show that $ab$ and $ba$ have the same order.

Claim: $|ab| = |ba|$
\begin{proof}
Let $|ab|=m$, $|ba|=n$

$(ab)^m = e$

$a^-m((ab)^m) = a^-m(e)$

$b^m = a^-m$

$b^{-m}(b^m) = b^{-m}(a^-m)$

$e = (ab)^{-m}$

$(ba)^m(e) = (ba)^m((ab)^{-m})$

$(ba)^m = e$ Since the order of $ba$ must be the least such exponent, $n \leq m$.

Alternatively,

$(ba)^n = e$

$b^{-n}(b^na^n) = b^{-n}(e)$

$a^n = b^{-n}$

$a^{-n}(a^n) = a^{-n}(b^{-n})$

$e = b^{-n}a^{-n}$

$e = (ab)^{-n}$

$(ab)^n = e$ Since the order of $ab$ must be the least such exponent, $m \leq n$.

Since, $m \leq n$ and $n \leq m$ then $m = n$.
\end{proof}
\end{question}
\end{question}

% --------------------------------------------------------------
%                         Q2
% --------------------------------------------------------------

\begin{question}{2}
Suppose $a$ and $b$ are elements of the group ($\Z_n$, +) that have the same order. Show
that $\langle a \rangle$ = $\langle b \rangle$.

Suppose $c \in \langle a \rangle, \langle b \rangle$. Since $|a|$ = $|b|$, This implies that $\langle a \rangle \subseteq \langle b \rangle$ and $\langle a \rangle \subseteq \langle b \rangle$, and therefore $\langle a \rangle$ = $\langle b \rangle$.

\end{question}

% --------------------------------------------------------------
%                         Q3
% --------------------------------------------------------------

\begin{question}{3} Let $g$ be an element of a multiplicative group $G$ with identity $e$. Show that if $g$ has order $n$ and $g^m = e$, then $n$ divides $m$.

If $m = qn + s : q,s \in \Z$ and $0 \leq s < n$. Then $g^s = g^{m-qn} = g^m(g^n)^{-q} = e(g^n)^{-q} = ee^{-q} = e$. Since $0 \leq s < n$, and $n$ is the smallest positive integer such that $g^m = e$, s must be 0, giving $m= qn$.

\end{question}

% --------------------------------------------------------------
%                         Q4
% --------------------------------------------------------------

\begin{question}{4}
Let $H = \{$
$ I =
\begin{bmatrix}
1 & 0 \\
0 & 1 \\
\end{bmatrix},
A = 
\begin{bmatrix}
 0 & -1 \\
-1 &  0 \\
\end{bmatrix},
B = 
\begin{bmatrix}
-1 &  0 \\
 0 & -1 \\
\end{bmatrix},
C = 
\begin{bmatrix}
0 & 1 \\
1 & 0 \\
\end{bmatrix}
\}$
\end{question}

\begin{question}{4a}
Make a table for the matrix multiplication operation on $H$.

\begin{tabular}[c]{c|cccc}
   & I & A & B & C  \\
   \hline
I  & I & A & B & C \\
A  & A & I & C & B \\
B  & B & C & I & A \\
C  & C & B & A & I
\end{tabular}

\end{question}

\begin{question}{4b}
Show that ($H$, $\times$) is isomorphic to the group ($U_8$, $\times$).
(Hint: Find an explicit bijection $\psi$: $H$ $\rightarrow$ $U_8$ that results in matching multiplication
tables.)

$U_n$ = $\{x \in \Z_n : gcd(x, n) = 1\}$ $|U_n| = \phi(n)$

$U_8$ = $\{1,3,5,7\}$
\\\\

\begin{tabular}[c]{c|cccc}
$H$ & I & A & B & C  \\
   \hline
I  & I & A & B & C \\
A  & A & I & C & B \\
B  & B & C & I & A \\
C  & C & B & A & I
\end{tabular}
$ \rightarrow \psi \rightarrow $
\begin{tabular}[c]{c|cccc}
$U_8$& 1 & 3 & 5 & 7  \\
   \hline
1  & 1 & 3 & 5 & 7 \\
3  & 3 & 1 & 7 & 5 \\
5  & 5 & 7 & 1 & 3 \\
7  & 7 & 5 & 3 & 1
\end{tabular}\\\\

\begin{tabular}{c|cccc}
x      & I & A & B & C \\
\hline
$\psi(x)$ & 1 & 3 & 5 & 7
\end{tabular}

\end{question}

% --------------------------------------------------------------
%                         Q5
% --------------------------------------------------------------

\begin{question}{5a}
Find the order of the linear groups ($GL_2$($\Z_5$), $\times$) and ($GL_2$($\Z_7$), $\times$)

$|GL_2$($\Z_p$)$| = (p^2-1)(p^2-p)$

$|GL_2$($\Z_5$)$| = (5^2-1)(5^2-5) = (24)(20) = 480$

$|GL_2$($\Z_7$)$| = (7^2-1)(7^2-7) = (48)(42) = 2016$

\end{question}

\begin{question}{5b}
Find the order of the symmetric groups $S_{10}$ and $S_{15}$.

$|S_n|$ = $n!$

$|S_{10}|$ = $10! = 3628800$

$|S_{15}|$ = $15! = 1307674400000$

\end{question}

% --------------------------------------------------------------
%                         Q6
% --------------------------------------------------------------

\begin{question}{6}
Let $f$ and $g$ be the elements of $S_7$ given by \\

$f= \begin{bmatrix}
1 & 2 & 3 & 4 & 5 & 6 & 7 \\
3 & 4 & 5 & 6 & 1 & 7 & 2
\end{bmatrix}
$\\

$g= \begin{bmatrix}
1 & 2 & 3 & 4 & 5 & 6 & 7 \\
5 & 1 & 7 & 3 & 2 & 4 & 6
\end{bmatrix}
$\\

Calculate $f^{-1}, f^2, f^3, g \circ f, and f^{-1} \circ g \circ f$.

$f^{-1}= \begin{bmatrix}
3 & 4 & 5 & 6 & 1 & 7 & 2 \\
1 & 2 & 3 & 4 & 5 & 6 & 7
\end{bmatrix}
$\\

$f^{2}= \begin{bmatrix}
1 & 2 & 3 & 4 & 5 & 6 & 7 \\
5 & 6 & 1 & 7 & 3 & 2 & 4
\end{bmatrix}
$\\

$f^{3}= \begin{bmatrix}
1 & 2 & 3 & 4 & 5 & 6 & 7 \\
1 & 7 & 3 & 2 & 5 & 4 & 6
\end{bmatrix}
$\\

$g \circ f = \begin{bmatrix}
1 & 2 & 3 & 4 & 5 & 6 & 7 \\
7 & 3 & 2 & 4 & 5 & 6 & 1
\end{bmatrix}
$\\

$f^{-1} \circ g \circ f = \begin{bmatrix}
1 & 2 & 3 & 4 & 5 & 6 & 7 \\
6 & 1 & 7 & 2 & 3 & 4 & 5
\end{bmatrix}
$\\
\end{question}

\end{document}